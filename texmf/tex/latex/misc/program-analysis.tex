% $Id$
%
% Program analysis notations, as in the book Principles of Program
% Analysis, by Flemming Nielson, Hanne Riis Nielson, and Chris Hankin
% (Springer-Verlag, 1999).

% Chapter 1

% Section 1.2
\newcommand{\AExp}{\textbf{AExp}}
\newcommand{\BExp}{\textbf{BExp}}
\newcommand{\Stmt}{\textbf{Stmt}}
\newcommand{\Var}{\textbf{Var}}
\newcommand{\Lab}{\textbf{Lab}}
\newcommand{\Num}{\textbf{Num}}
\newcommand{\Opa}{\ensuremath{\textbf{Op}_a}}
\newcommand{\Opb}{\ensuremath{\textbf{Op}_b}}
\newcommand{\Opr}{\ensuremath{\textbf{Op}_r}}
\newcommand{\opa}{\ensuremath{\textit{op}_a}}
\newcommand{\opb}{\ensuremath{\textit{op}_b}}
\newcommand{\opr}{\ensuremath{\textit{op}_r}}

% programs

% The macros that don't start with ``L'' for elementary blocks do
% their own counting of the labels for you.
% Use \RESETLABELS to start over (from 1) again.
% Those starting with L you have to label yourself.
\newcounter{PALABEL}
\setcounter{PALABEL}{0}
\newcommand{\RESETLABELS}{\setcounter{PALABEL}{0}}
\newcommand{\NEXTLABEL}{{\stepcounter{PALABEL}\thePALABEL}}
% Elementary blocks
\newcommand{\LASSIGN}[3]{\textrm{[}{#1} := {#2}\textrm{]}\ensuremath{^{#3}}}
\newcommand{\ASSIGN}[2]{\LASSIGN{#1}{#2}{\NEXTLABEL}}
\newcommand{\LSKIP}[1]{\textrm{[}\texttt{skip}\textrm{]}\ensuremath{^{#1}}}
\newcommand{\SKIP}{\LSKIP{\NEXTLABEL}}
\newcommand{\LEXP}[2]{\textrm{[}#1\textrm{]}\ensuremath{^{#2}}}
\newcommand{\EXP}[1]{\LEXP{#1}{\NEXTLABEL}}

%% programs
\makeatletter
\@ifundefined{myobeycr}{%{obey}

% \myobeycr is same as \obeycr, but doesn't do a \@gobblecr.
{\catcode`\^^M=13 \gdef\myobeycr{\catcode`\^^M=13 \def^^M{\\}}%
\gdef\restorecr{\catcode`\^^M=5 }}

% \obeytabs
{\catcode`\^^I=13 \gdef\obeytabs{\catcode`\^^I=13 \def^^I{\hbox{\hskip 4em}}}}

% \obeyspaces
{\obeyspaces\gdef {\hbox{\hskip0.5em}}}

% environment for displayed text, indented, obeys cr, tab, spaces
\newenvironment{obeyDisplay}{%
\samepage%
\begin{list}{}{}\item\obeyspaces\obeytabs\obeycr}{%
\end{list}}
}{\relax}
\makeatother

% one can renew this command to use a different font for program text
\newcommand*{\programFontDeclaration}{\normalfont\ttfamily}

% environment for programs, indented, set in typewriter font, obeys cr, tab,etc
\newenvironment{program*}{%
\samepage\programFontDeclaration\RESETLABELS%
\begin{list}{}{}\item\obeyspaces\obeytabs\obeycr}{%
\end{list}}

% Section 1.3
\newcommand{\EXIT}[1]{\ensuremath{{#1}_{\textit{\small exit}}}}
\newcommand{\ENTRY}[1]{\ensuremath{{#1}_{\textit{\small entry}}}}
\newcommand{\ANALYSIS}[1]{\textsf{#1}}
\newcommand{\ANALYSISSUB}[1]{\textsf{\scriptsize #1}}
\newcommand{\RD}{\ANALYSIS{RD}}
\newcommand{\RDENTRY}{\ENTRY{\RD}}
\newcommand{\RDEXIT}{\EXIT{\RD}}

% constraint based analysis, section 1.4
\newcommand{\C}{\ensuremath{\widehat{C}}}
\newcommand{\ENV}{\ensuremath{\widehat{\rho}}}
\newcommand{\LFN}[3]{\texttt{fn #1 => } \LEXP{#2}{#3}}
\newcommand{\FN}[2]{\LFN{#1}{#2}{\NEXTLABEL}}

% math definitions
\newcommand{\implies}{\ensuremath{\Rightarrow}}
\newcommand{\impliedby}{\ensuremath{\Leftarrow}}
\newcommand{\IFF}{\Leftrightarrow}
\newcommand{\arr}{\ensuremath{\rightarrow}}
\newcommand{\DOM}{\ANALYSIS{DOM}}
\newcommand{\POWERSET}[1]{\ensuremath{{\cal P}}(#1)}

% abstract interpretation, section 1.5
\newcommand{\CS}{\ANALYSIS{CS}}
\newcommand{\CSENTRY}{\ENTRY{\CS}}
\newcommand{\CSEXIT}{\EXIT{\CS}}
\newcommand{\SRD}{\ANALYSIS{SRD}}

% Annotated Type constructors
\newcommand{\RULE}[1]{[\textit{#1}]}
\newcommand{\ANNARR}[2]{\ensuremath{-\!\!\!\frac{#1}{#2}\!\!\!\arr}}
\newcommand{\Sarr}[2]{\ensuremath{\Sigma \ANNARR{#1}{#2} \Sigma}}

% Chapter 2

% Section 2.1
% Flow graphs, p. 34ff
\newcommand{\INIT}{\textsl{init}}
\newcommand{\FINAL}{\textsl{final}}
\newcommand{\FLOW}{\textsl{flow}}
\newcommand{\FLOWR}{\ensuremath{\textsl{flow}^{R}}}
\newcommand{\BLOCKS}{\textsl{blocks}}
\newcommand{\LABELS}{\textsl{labels}}

% various sets, p. 36ff
\newcommand{\STAR}[1]{\ensuremath{{#1}_{\star}}}
\newcommand{\Block}{\Blocks}
\newcommand{\Blocks}{\textbf{Blocks}}
\newcommand{\ProgLab}{\STAR{\Lab}}
\newcommand{\ProgVar}{\STAR{\Var}}
\newcommand{\ProgBlock}{\STAR{\Block}}
\newcommand{\ProgAExp}{\STAR{\AExp}}
\newcommand{\Prog}{\STAR{S}}
\newcommand{\FV}{\textsl{FV}}

\newcommand{\KILL}[1]{\ensuremath{\textsl{kill}_{\ANALYSISSUB{#1}}}}
\newcommand{\GEN}[1]{\ensuremath{\textsl{gen}_{\ANALYSISSUB{#1}}}}

% 2.1.1 Available Expressions
\renewcommand{\AE}{\ANALYSIS{AE}}
\newcommand{\AEENTRY}{\ENTRY{\AE}}
\newcommand{\AEEXIT}{\EXIT{\AE}}
\newcommand{\KILLAE}{\KILL{AE}}
\newcommand{\GENAE}{\GEN{AE}}
% 2.1.2 Reaching Definitions
\newcommand{\KILLRD}{\KILL{RD}}
\newcommand{\GENRD}{\GEN{RD}}
% 2.1.3 Very Busy Expressions
\newcommand{\VB}{\ANALYSIS{VB}}
\newcommand{\VBENTRY}{\ENTRY{\VB}}
\newcommand{\VBEXIT}{\EXIT{\VB}}
\newcommand{\KILLVB}{\KILL{VB}}
\newcommand{\GENVB}{\GEN{VB}}
% 2.1.4 Live Variables
\newcommand{\LV}{\ANALYSIS{LV}}
\newcommand{\LVENTRY}{\ENTRY{\LV}}
\newcommand{\LVEXIT}{\EXIT{\LV}}
\newcommand{\KILLLV}{\KILL{LV}}
\newcommand{\GENLV}{\GEN{LV}}

% 2.1.5 Derived Data Flow Information
\newcommand{\ud}{\textsl{ud}}
\newcommand{\CLEAR}{\textsl{clear}}
\newcommand{\USE}{\textsl{use}}
\newcommand{\DEF}{\textsl{def}}
\newcommand{\du}{\textsl{du}}
\newcommand{\UD}{\ANALYSIS{UD}}
\newcommand{\DU}{\ANALYSIS{DU}}

% 2.2.1 Structural Operational Semantics
\newcommand{\State}{\textbf{State}}
\newcommand{\s}{\ensuremath{\sigma}}
\newcommand{\CONFIG}[1]{\ensuremath{\langle #1 \rangle}}
\newcommand{\REDUCESTO}{\rightarrow}
\newcommand{\WITHFOR}[3]{\ensuremath{#1[#2\mapsto #3]}} % subst #2 for #3 in #1
% brackets that surround syntax (as in denotational semantics)
\newcommand{\synbracket}[1]{[\![#1]\!]}
\newcommand{\MEANING}[2]{\ensuremath{{\cal #1}}\synbracket{#2}}
\newcommand{\DENAEXP}[1]{\MEANING{A}{#1}}
\newcommand{\DENBEXP}[1]{\MEANING{B}{#1}}
\newcommand{\DENOP}[1]{\MEANING{O}{#1}}

\newcommand{\INTEGER}{\textbf{Z}}
\newcommand{\BOOLEAN}{\textbf{T}}
\newcommand{\TRUE}{\textsl{true}}
\newcommand{\FALSE}{\textsl{false}}

% erratic choice operator []
\newcommand{\ERRATIC}{\framebox[0.45em]{\rule{0em}{0.5em}}}

% 2.2.2 Theoretical Properties
\newcommand{\EQUATIONS}[1]{\ensuremath{\ANALYSIS{#1}^{=}}}
\newcommand{\CONSTRAINTS}[1]{\ensuremath{\ANALYSIS{#1}^{\subseteq}}}
\newcommand{\GENERATOR}[1]{\ensuremath{F^{\Prog}_{\ANALYSISSUB{#1}}}}
\newcommand{\GENERATORFOR}[2]{\ensuremath{F^{#2}_{\ANALYSISSUB{#1}}}}
\newcommand{\LIVE}{\textsl{live}}
\newcommand{\REACH}{\textsl{reach}}
\newcommand{\entry}{\textsl{entry}}
\newcommand{\exit}{\textsl{exit}}
% fixed points
\newcommand{\LFP}{\textsl{lfp}}
\newcommand{\GFP}{\textsl{gfp}}
\newcommand{\REDUCTIVE}{\textsl{Red}}
% lattice theory
\newcommand{\meet}{\ensuremath{\sqcap}}
\newcommand{\join}{\ensuremath{\sqcup}}
% \bigmeet is a hack, because latex doesn't have \bigsqcap
\newcommand{\bigmeet}{\meet}
%% a better version of \bigmeet is in the stmaryrd package.  To use it do:
%  \usepackage{stmaryrd}
%  \renewcommand{\bigmeet}{\bigsqcap}
\newcommand{\bigjoin}{\bigsqcup}

\newcommand{\LAMBDA}[1]{\lambda{#1}\:.\:}

% 2.3 Monotone frameworks
\newcommand{\FUNS}{{\cal F}}
\newcommand{\ANALYSISo}[1]{\ensuremath{\ANALYSIS{#1}_{\circ}}}
\newcommand{\ANALYSISdot}[1]{\ensuremath{\ANALYSIS{#1}_{\bullet}}}

% example 2.36 and exercise 2.14
\newcommand{\SIGN}{\textbf{Sign}}
\newcommand{\SUBSIGN}[1]{\ensuremath{#1_{\textsf{sign}}}}
\newcommand{\SUPSIGN}[1]{\ensuremath{#1^{\textsf{sign}}}}
\newcommand{\LSIGN}{\SUBSIGN{L}}
\newcommand{\ASIGNOF}[1]{\SUBSIGN{{\cal A}}\synbracket{#1}}
\newcommand{\OPSIGNOF}[1]{\SUBSIGN{{\cal O}}\synbracket{#1}}
\newcommand{\ssign}{\SUPSIGN{\sigma}}
\newcommand{\fsign}{\SUPSIGN{f}}
\newcommand{\phisign}{\SUPSIGN{\phi}}
\newcommand{\sign}{\textsl{sign}}
\newcommand{\PLUS}{\texttt{+}}
\newcommand{\MINUS}{\texttt{-}}
\newcommand{\ZERO}{\texttt{0}}

% Section 2.5 Interpreocedural Analysis
\newcommand{\ProgDecl}{\STAR{D}}
\newcommand{\INTERFLOW}{\textsl{inter-flow}}
% use \STAR{\FLOW} etc for the starred versions.
\newcommand{\LIS}[1]{\texttt{is}\ensuremath{^{#1}}}
\newcommand{\IS}{\LIS{\NEXTLABEL}}
\newcommand{\LEND}[1]{\texttt{end}\ensuremath{^{#1}}}
\newcommand{\END}{\LEND{\NEXTLABEL}}
\newcommand{\LCALL}[3]{\textrm{[}\texttt{call }#1\textrm{]}\ensuremath{^{#2}_{#3}}}
\newcommand{\CALL}[1]{\LCALL{#1}{\NEXTLABEL}{\NEXTLABEL}}
\newcommand{\PROCSKEL}{\texttt{proc}~p\/(\texttt{val}~x, \texttt{res}~y)~\LIS{\ell_n}~S~\LEND{\ell_x}}

% Section 2.5.1 SOS for procedures
\newcommand{\loc}{\ensuremath{\xi}}
\newcommand{\Loc}{\textbf{Loc}}
\newcommand{\env}{\ensuremath{\rho}}
\newcommand{\Env}{\textbf{Env}}
\newcommand{\store}{\ensuremath{\varsigma}}
\newcommand{\Store}{\textbf{Store}}
\newcommand{\finarr}{\ensuremath{\rightarrow_{\rm fin}}}

% Section 2.5.2 Intra vs Interprocedural Analysis
\newcommand{\EMBELLISH}[1]{\ensuremath{\widehat{#1}}}
\newcommand{\HATL}{\EMBELLISH{L}}
\newcommand{\hatf}[1]{\EMBELLISH{f_{#1}}}
\newcommand{\hatfone}[1]{\EMBELLISH{f^1_{#1}}}
\newcommand{\hatftwo}[1]{\EMBELLISH{f^2_{#1}}}
\newcommand{\ISCALL}{\textsl{call}}
\newcommand{\ISRETURN}{\textsl{return}}
\newcommand{\POP}{\textsl{pop}}
\newcommand{\FORMALS}{\textsl{formals}}
\newcommand{\VALFORMALS}{\textsl{val-formals}}
\newcommand{\RESFORMALS}{\textsl{res-formals}}
\newcommand{\ACTUALS}{\textsl{actuals}}
\newcommand{\VALACTUALS}{\textsl{val-actuals}}
\newcommand{\RESACTUALS}{\textsl{res-actuals}}

% Section 2.6 Shape Analysis
\newcommand{\sel}{\textit{sel}}
\newcommand{\Sel}{\textbf{Sel}}
\newcommand{\p}{\textit{p}}
\newcommand{\PExp}{\textbf{PExp}}
% \loc and \Loc are above
\newcommand{\nil}{\textbf{\diamond}}
\newcommand{\Heap}{\textbf{Heap}}
\newcommand{\h}{{\cal \scriptstyle H}}
\newcommand{\DENP}[1]{\MEANING{P}{#1}}
% \finarr is above

% Section 2.6.2, shape graphs
\newcommand{\AS}{\textsf{S}}
\newcommand{\AH}{\textsf{H}}
\newcommand{\is}{\textsf{is}}
\newcommand{\ALoc}{\textbf{ALoc}}
\newcommand{\AState}{\textbf{AState}}
\newcommand{\AHeap}{\textbf{AState}}
\newcommand{\SG}{\textbf{SG}}
\newcommand{\Shape}{\textsl{Shape}}

% Chapter 4
\newcommand{\CORRECTNESSFOR}[1]{\ensuremath{R_\ANALYSISSUB{#1}}}
\newcommand{\ABSTRFOR}[1]{\ensuremath{\beta_\ANALYSISSUB{#1}}}
% the following require the latexsym package
% ~> is \leadsto
% |> is \rhd
