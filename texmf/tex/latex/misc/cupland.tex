%% LaTeX2e file `cupland.tex'
%% generated by the `filecontents' environment
%% from source `FoCBS' on 1999/06/21.
%%
%%%%%%%%%%%%%%%%%%%%%% start of cupland.tex %%%%%%%%%%%%%%%%%%%%%%
% cupland.tex (for cup6)
% v1.1 --- released 11th February 1993, changed by M. Reed
% v1.0 --- first release 2nd November 1992

\documentstyle[landscape,cup6a]{cupbook}

\begin{document}

This is a landscape page which will be much wider but shorter than the
standard portrait page. Such pages should only be used for excessively
wide floats which do not fit the standard (portrait) measure.

\setcounter{chapter}{3}
\setcounter{table}{0}
\begin{table*}
 \caption{The Largest Optical Telescopes}
 \label{tab1}
\begin{tabular}{lccccc}
 \hline \hline
      &           &              &            &  Diameter of & \\
 Site & Ownership & Altitude (m) & Hemisphere &  primary (m) & Year\\
 \hline
 Mt Pastoukhow, Caucasus, USSR   & USSR & 2050 & N & 6.05 & 1974\\
 Mt Palomar, California, USA     & USA  & 1800 & N & 5.08 & 1949\\
 Mt Hopkins, Arizona, USA        & USA  & 2600 & N & \phantom{$^a$}4.60%
  \footnote{Multiple mirror telescope equivalent
                                  light collecting area.} & 1979\\
 La Palma, Canary Islands, Spain & UK & 2400 & N & 4.20 & 1986\\
 Kit Peak, Arizona, USA          & USA  & 2100 & N & 4.01 & 1973\\
 Cerro Tololo, Chile             & USA  & 2500 & S & 4.01 & 1974\\
 Siding Spring, New South Wales,
    Australia         & UK--Australia   & 1200 & S & 3.88 & 1974\\
 Mauna Kea, Hawaii, USA          &  UK  & 4200 & N & 3.80 & 1979\\
 Mauna Kea, Hawaii, USA          &
                      Canada--France--  & 4200 & N & 3.60 & 1979\\
                               & Hawaii &      &   &      &     \\
 La Silla, Chile               & Europe & 2450 & S & 3.57 & 1976\\
 Calar Alto, Spain         & FRG--Spain & 2160 & N & 3.50 & 1979\\
 Mt Hamilton, California, USA    & USA  & 1300 & N & 3.05 & 1959\\
 Mauna Kea, Hawaii, USA          & USA  & 4200 & N & 3.00 & 1979\\
 \hline \hline
\end{tabular}
%
\end{table*}

\clearpage
\setcounter{figure}{12}
\begin{figure*}
 \vspace{9pc}
 -- Landscape illustration --
 \vspace{9pc}
 \caption{Chart for a cold fully ionized, loss free proton plasma with
          \protect $\omega_0/\omega_{\rm H} = 2.0\protect $.
 The upper and lower strips refer to the Ordinary and Extraordinary
  waves respectively, and the shaded parts show where these waves are
  cut off (evanescent) for all real wave normal and ray directions.
 The transition frequencies marked at the top are explained in the text.
 The  frequency scale is divided into sections and is linear in each section.
 The horizontal lines show the frequency ranges where the various
  features of the field, as indicated at the left, are present.
 The numbers above and below them give the values in degrees of
  $\alpha$ and $\upsilon$, respectively, at the ends of the ranges.}
 \label{fig13}
\end{figure*}

\end{document}
%%%%%%%%%%%%%%%%%%%%%% end of cupland.tex %%%%%%%%%%%%%%%%%%%%%%
