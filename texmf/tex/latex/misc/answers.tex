\newcommand{\IGNOREANSWER}[1]{\vfill}
\newcommand{\PRINTANSWER}{\par Answer:~~}
\newcommand{\IGNOREANSWERTEXT}[1]{\relax}
\newcommand{\PRINTANSWERTEXT}{\relax}

\newcommand{\IGNOREDIRECTIONS}[1]{\relax}
\newcommand{\PRINTDIRECTIONS}{\relax}
\newcommand{\ANSWERRULE}[1]{\rule{#1}{0.1mm}}

% the following asks whether you want answers when you run latex...
\typein[\USEANSWERS]{Do you want answers included (y/n)?}
\if n\USEANSWERS \newcommand{\ANSWER}[1]{\IGNOREANSWER{#1}}
    \newcommand{\ANSWERTEXT}[1]{\IGNOREANSWERTEXT{#1}}
    \newcommand{\DIRECTIONS}{\PRINTDIRECTIONS}
    \newcommand{\BLANKFORANSWER}[2]{\ANSWERRULE{#1}}
\else \newcommand{\ANSWER}{\PRINTANSWER}
    \newcommand{\ANSWERTEXT}{\PRINTANSWERTEXT}
    \newcommand{\DIRECTIONS}[1]{\IGNOREDIRECTIONS{#1}}
    \newcommand{\BLANKFORANSWER}[2]{\makebox[#1]{{\it #2\/}}}
\fi

% the following commands use the verbfile style
% so there must be \documentstyle[verbfile]{article} at the top of the latex
% file.


\newcommand{\VERBFILE}[1]{
 \begin{list}{}{\listparindent 0.0in \leftmargin 0.0in \labelwidth 0.0in
                \labelsep 0.0in \itemindent -2.5em}
 \item
 \verbfile{#1}
 \end{list}
}
\newcommand{\INPUTCODE}[1]{
 The file {\tt #1} contains the following code.
 \VERBFILE{#1}
}
\newcommand{\INPUTTRANSCRIPT}[1]{
 Testing of the above code is shown in the following transcript.
 \VERBFILE{#1}
}
