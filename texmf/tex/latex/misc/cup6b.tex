%% LaTeX2e file `cup6b.tex'
%% generated by the `filecontents' environment
%% from source `FoCBS' on 1999/06/21.
%%
%%%%%%%%%%%%%%%%%%%%%% start of cup6b.tex %%%%%%%%%%%%%%%%%%%%%%
% cup6b.tex
% v1.1 --- released 15th February 1993, changed by M. Reed
% v1.0 --- first release 2nd November 1992

\documentstyle[cup6b]{cupbook}

\begin{document}
%
\title{THE SCIENCE OF \\TECHNOLOGICAL MEDICINE}
\halftitle{THE SCIENCE OF TECHNOLOGICAL MEDICINE}
\author{A. N. EDITOR\\
        Head Reader, Quicksilver College, Wayoutsville}
\date{10 August 1992}

\newtheorem{theorem}{Theorem}[chapter]
\newcommand{\blackboard}{\bf }

\hyphenation{tele-vision}
\hyphenation{spermiogen-esis sperm-iogenesis}

\pagenumbering{roman}
\maketitle % this \inputs \jobname.ttl if available

\chapter*{}
\thispagestyle{empty}
\begin{center}
  To my ever loving
\end{center}

\cleardoublepage
\tableofcontents
\cleardoublepage

\begin{listofcontributors}
  \item [D. G. Armour]  Department of Electronic and Electrical
Engineering, University of Salford, Salford M5 4WT, UK
  \item [H. E. Bishop]  Materials Development Division, AERE Harwell,
Didcot, Oxon OX11 0RA, UK
  \item [A. B. Christie]  VG Scientific, Imberhorne Lane, East
Grinstead, West Sussex RH19 1UB, UK
  \item [W. A. Grant]  Department of Electronic and Electrical
Engineering, University of Salford, Salford M5 4WT, UK
  \item [M. P. Seah]  Division of Materials and Applications, National
Physical Laboratory, Teddington, Middlesex TW11 0LW, UK
  \item [R. Smith]  Loughborough University of Technology, Loughborough,
Leicestershire LE11 3TU, UK
  \item [D. E. Sykes]  Loughborough University of Technology,
Loughborough, Leicestershire LE11 3TU, UK
  \item [J. C. Vickerman]  Surface Analysis Research Centre, University
of Manchester, Institute of Science and Technology, Manchester M60 1QD, UK
  \item [J. M. Walls]  VG Ionex, The Maltings, Burgess Hill, West
Sussex RH15 9TQ, UK
 \end{listofcontributors}

\cleardoublepage
\chapter*{Preface}

\begin{center}
 \mbox{}\\[-6.5pt]
 \begin{minipage}{12pc}
   \begin{verse}
    They expounded the reazles\\
    For sneezles\\
    And wheezles,\\
    The manner of measles\\
    When new.\\
    They said `If he freezles\\
    In draughts and in breezles,\\
    Then PHTHEEZLES\\
    May even ensue.'
  \end{verse}
 \end{minipage}
\end{center}
\begin{quote}\centering
 \vspace{-4.3pt}
 {\it (A. A. Milne: Now We Are Six)\phantom{ix))i}}
\end{quote}
Symmetry is like a disease. Or, perhaps more accurately, it
{\it is\/} a disease. At least in my case; I seem to have
a bad case of it. Let me tell you how this came about.

I must always have had a tendency to symmetry. An early mild
symptom was a special liking for series of similar things:
pads of paper, piles of filing cards, sets of pencils and crayons.
My drawings and doodlings inclined to periodicity. Though a lover
of serious music, I have long had a special place in my heart for
marches (with their strict rhythm). But things took a turn for the
worse when I started to work toward my Ph.D. degree. I had decided
to do my thesis research in the field of what was then called
theoretical elementary particle physics. (Since the question of
elementarity is an open one, we now prefer to call the field
`high energy physics' or `physics of particles
and fields'.) My thesis advisor introduced me to this field
through the study of group theory and its application to elementary
particle symmetries, and I finished my Ph.D.  with a thesis entitled
`Several aspects of particle symmetries and their origins'.
\begin{flushright}\it
  A. N. Editor
\end{flushright}

\cleardoublepage
\pagenumbering{arabic}
\part{Starting} % part 1

\author[J. M. Walls]{J. M. WALLS\\Affiliation}
\chapter{Methods of surface analysis} % chap 1

\author[R. Smith and J. M. Walls]
       {R. SMITH \\ Affiliation of R. Smith \and J.\,M. WALLS \\
        Affiliation of J. M. Walls}
\chapter{Ion erosion in surface analysis} % chap 2

\part{Physical exertion} % part 2

\author[M. P. Seah]{M. P. SEAH\\University of Cambridge}
\chapter{Electron and ion energy analysis} % chap 3
%\setcounter{page}{3}
\setcounter{equation}{7}

\begin{abstract}
 The use of pharmacological agents offers an important option in the
development of a male antifertility method using spermatids as the
targets. Compounds that can pass through the blood-testis barrier to
reach the spermatids may have some chance of interfering with the
spermiogenesis. Possible compounds are those that are inhibitory to the
key events in spermiogenesis, namely acrosomal glycoprotein synthesis,
energy generation from lactate and axoneme formation. It may also be
possible to identify a temperature-sensitive event in the spermatids
and to interfere with it using a chemical agent.
 \end{abstract}
 For direct photography, with any telescope, it is particularly
important that the detector have no direct view of the sky. Direct sky
light is normally intercepted by baffles, and these are shown in the
sectional view of the ST optics in Fig. 3.35. The need for, and the
extent of, baffling usually restrict the size of the field which can
receive light from the whole primary (unvignetted field) (see Table
\ref{tab1}).
\begin{table*}
 \caption{The largest optical telescopes}
 \label{tab1}
 \vspace*{500pt}
\end{table*}

For ground based, multi-purpose telescopes, \mbox{Ritchey--Chr\'etien}
systems have the disadvantage that coma is increased at all foci other
than that for which the system was optimised (see Fig. \ref{fig1}).
\begin{figure}
 \vspace{350pt}
 \caption{({\it a\/}) The expected point spread function of the Hubble Space
              Telescope, and the distribution of encircled energy in
              the image of \mbox{623.8 nm.}
 ({\it b\/}) The variation of image quality. Reproduced from Alexander (1985).}
 \label{fig1}
\end{figure}
Although a Cassegrain RC telescope cannot be used for prime focus wide
field photography without correctors, the comatic correction in the
primary means that the correctors are smaller and simpler (usually
two components) than those necessary to correct to a wide field for
a paraboloid. Generally the finding field required at the coud\'e focus
for on-axis spectroscopy is about 1 or 2 arc minutes which can be
achieved without correction for an RC Cassegrain.

\section{Instrument adaptors/offset guiders}

For most large optical telescopes certain basic functions which
are common to several instruments are supplied in an adaptor or guider
which acts as the mechanical interface to the telescope (see Fig. \ref{fig2}).
\begin{figure}
 \vspace{250pt}
 \caption{({\it a\/}) The expected point spread function of the Hubble Space
              Telescope, and the distribution of encircled energy in
              the image of \mbox{623.8 nm.}
 ({\it b\/}) The variation of image quality. Reproduced from Anderson {\it et al\/} (1989).}
 \label{fig2}
\end{figure}
It provides rotation and defines the position of the focal plane. The
adaptor allows the observer to examine an object or star field before
starting an exposure by providing a full-field view within an eyepiece
or low light level television system and by acquiring a guide star for
use during the exposure. The adaptor often includes calibration sources
and the optics necessary to feed light from them into the auxiliary
equipment (see Table \ref{tab2}).
\begin{table*}
 \caption{Characteristics of the Hubble Space Telescope}
 \label{tab2}
\begin{tabular}{@{\hspace{1.75pc}}ll} % 3.5pc space added to centre the table
\hline
\hline
Aperture                             & $\rm \, 2.4 m $ \\
Focal ratio                          & $ f/24 $ \\
Maximum obscuration ratio (linear)   & $ 0.34 $ \\
Effective focal length               & $ \rm 57.6\, m $ \\
Back focal length (BFL)              & $ \rm 1.5 \, m $ \\
Primary to secondary spacing         & $ \rm 4.90\, m $ \\
Plate scale                          & $ \rm 3.58\, arcsec\ mm^{-1}$ \\
Field of view diameter               & $ \rm 28\, arcmin\ (467\, mm)$ \\
Data field diameter                  & $ \rm 18\, arcmin\ (300\, mm)$ \\
Tracking field size                  & $ \rm 1.5 \times 10^{-5} sr
                                          \  (180\, arcmin)^2$\\
Mirror coating                       & Al coated with $ \rm MgF_2$\\
Primary mirror focal ratio           & $ f/2.3 $\\
Secondary mirror focal ratio         & $ f/2.23 $ \\
Secondary mirror aperture            & $ \rm 0.31\, m $\\
Secondary mirror magnification       & $ 10.4 $\\
\hline
\hline
\end{tabular}
\end{table*}

Figure 3.37 shows a schematic of a simple adaptor used at the prime
focus of the CFH 3.6 m telescope. The field acquisition mirror
allows the field to be verified and can be swung out of the way during
an exposure. A small probe mirror is used in the focal plane outside
the area of the plate to pick up a guide star.

For most large optical telescopes certain basic functions
which are common to several instruments are supplied in an adaptor or
guider which acts as the mechanical interface to the telescope. It
provides rotation and defines the position of the focal plane. The
adaptor allows the observer to examine an object or star field before
starting an exposure by providing a full-field view within an eyepiece
or low light level television system and by acquiring a guide star for
use during the exposure. The adaptor often includes calibration sources
and the optics necessary to feed light from them into the auxiliary
equipment.

\section{Infrared telescopes}

For most large optical telescopes certain basic functions which
are common to several instruments are supplied in an adaptor or guider
which acts as the mechanical interface to the telescope. It provides
rotation and defines the position of the focal plane. The adaptor allows
the observer to examine an object or star field before starting an
exposure by providing a full-field view within an eyepiece or low light
level television system and by acquiring a guide star for use during
the exposure. The adaptor often includes calibration sources and the
optics necessary to feed light from them into the auxiliary equipment.

For most large optical telescopes certain basic functions which are
common to several instruments are supplied in an adaptor or guider which
acts as the mechanical interface to the telescope. It provides rotation
and defines the position of the focal plane. The adaptor allows the
observer to examine an object or star field before starting an exposure
by providing a full-field view within an eyepiece or low light level
television system and by acquiring a guide star for use during the
exposure. The adaptor often includes calibration sources and the optics
necessary to feed light from them into the auxiliary equipment.

For most large optical telescopes certain basic functions which are
common to several instruments are supplied in an adaptor or guider which
acts as the mechanical interface to the telescope. It provides rotation
and defines the position of the focal plane. The adaptor allows the
observer to examine an object or star field before starting an exposure
by providing a full-field view within an eyepiece or low light level
television system and by acquiring a guide star for use during the
exposure. The adaptor often includes calibration sources and the optics
necessary to feed light from them into the auxiliary equipment.

The maximum of the thermal emission at \mbox{300 K} is near $\rm 10\,
\mu m$. As a result, at wavelengths $\rm > 2.5\,  \mu m$, thermal
radiation from the sky and from any structures included within the
telescope beam contributes a high background signal. Telescopes
designed for optical observations are far from optimal for use in the
infrared because of the following features (Barrow, 1984):
\begin{enumerate}[(iii)]\listsize
 \item the Cassegrain focal ratio is too small,
 \item direct view of the sky is blocked by baffles and all visible
       surfaces apart from the mirrors are blackened,
 \item the unvignetted field of view is large and corrected,
 \item mirror surfaces are coated with aluminium.
\end{enumerate}
Infrared detectors must be operated at low temperatures which
means that the telescope entrance pupil should subtend as small a
solid angle as possible at the detector in order to maintain proper
refrigeration. This condition dictates a focal ratio  $ > 20$.

This happens where $\nu_{\hbox{\sc t}}$ is the effective thermal
spread of the beam in units of the thermal spread of the stationary
plasma $( = (\phi_1 - \gamma_2) / {2 \phi_3})$.
We
restrict attention to weakly unstable situations by defining a critical
value of velocity $\nu_{\hbox{\scriptsize c}}$ such that $D$ and
$\partial D / \partial \omega$ are zero for $ \nu = \nu_{\hbox{\scriptsize c}}$.
For values of $\nu$ near $\nu_{\hbox{\scriptsize c}}$
we have $\omega = \omega_{\hbox{\scriptsize e}} + {\rm i} \alpha$ where
\begin{equation}
 \alpha^2 = \frac{2 ( \nu - \nu_{\hbox{\scriptsize c}})(\partial D / \partial \nu_{\hbox{\scriptsize c}})}
                 {\partial^2 D/ \partial \omega^2}.
\end{equation}
Thus for $ \nu > \nu_{\hbox{\scriptsize c}} $ the system
is unstable. For values of $ \nu \simeq \nu_{\hbox{\scriptsize c}}$,
the growth rate $\alpha$ is much smaller than $\omega_{\hbox{\scriptsize c}}$
so we have a situation of two distinct time scales (see Fig. \ref{fig3}).
 \begin{figure} % this is figure 3.3.
  \vspace*{2.5in}
  \caption{({\it a\/}) The expected point spread function of the Hubble Space
               Telescope, and the distribution of encircled energy in
               the image of \mbox{623.8 nm.}
  ({\it b\/}) The variation of image quality. Reproduced from Zabreki (1986).}
 \label{fig3}
\end{figure}
This is a situation where a nonlinear theory can be formulated using
the multiple time perturbation method.\footnote{We
may interchange the sequence of the
transformations: if we first rotate and then dilate, the same
points are obtained in the end.}
Such a method
is discussed in detail in Chapter 5 and was applied to the present
problem by El-Labany \& Rowlands (1986). In this way one finds
\begin{equation}
  {\mit \Delta}\phi_1 = \phi_1 - \phi_1
   = B_1 \rho(t) \sin \psi - B_3 r^3(t) \sin 3 \psi + O(\rho^5 ).
 \label{eq:explicit}
\end{equation}
where $\psi = \omega_{\hbox{\scriptsize p}}t - kx - \sigma(t)$.
$B_1$ and $B_3$ are constants and
depend on the plasma parameters $\nu_{\hbox{c}}$ and $\delta$.
The quantities $\rho$ and $\sigma$  are related to the
amplitude of the electric field by $E=\rho\exp({\rm i}\sigma)$ and
\begin{equation}
\frac{{\rm d}^2 E}{{\rm d} t^2} - \alpha^2 E -
   b \left| E \right| ^2 = 0.
\end{equation}
Here $ \alpha $ is the linear growth rate given above whilst
$b $ is a function of $\nu _{\hbox{c}}$, $\delta$ and
$k $. The above constitutes an approximate solution to the two
stream problem, limited in amplitude because of neglect of terms of
order $\rho^5$ and is the analogue of the solution (3.1)
to the cold plasma problem.

Now we introduce a Lagrangian type formulation by replacing the
explicit solution (\ref{eq:explicit}) by the implicit one
\begin{equation}
\Delta \phi_1 = A \sin \overline{\psi} + B \sin 2 \overline{\psi},
\end{equation}
where
\begin{equation}
 \psi = \overline{\psi} + \beta \sin \overline{\psi}.
 \label{eq:psibar}
\end{equation}
The quantities $A$, $B$ and $\beta$ are now determined
by expanding (\ref{eq:psibar})
in powers of $\beta$ (essentially in powers of
$\rho$) and equating powers of $\sin (n \psi)$ for
$n = 1$, $2$ and $3$. In this way we find in analogy
with (3.2) that
\begin{displaymath}
\beta^2 = 3 B_3 \rho^2/B_1, \; A = (B_1 - 3 B_3\rho^2)\rho
\end{displaymath}
and
\begin{displaymath}
B = \rho^2 \sqrt{2 B_1 B_3},
\end{displaymath}
here leaving this equation incomplete.

For direct photography, with any telescope, it is particularly
important that the detector have no direct view of the sky (see
Fig. \ref{fig13}).
\setcounter{figure}{12}
\begin{figure*}
 \vspace{200pt}
 -- Landscape illustration --
 \vspace{200pt}
 \caption{Chart for a cold fully ionized, loss free proton plasma with
          \protect $\omega_0/\omega_{\rm H} = 2.0\protect $.
 The upper and lower strips refer to the Ordinary and Extraordinary
  waves respectively, and the shaded parts show where these waves are
  cut off (evanescent) for all real wave normal and ray directions.
 The transition frequencies marked at the top are explained in the text.
 The  frequency scale is divided into sections and is linear in each section.
 The horizontal lines show the frequency ranges where the various
  features of the field, as indicated at the left, are present.
 The numbers above and below them give the values in degrees of
  $\alpha$ and $\theta$, respectively, at the ends of the ranges.}
 \label{fig13}
\end{figure*}
Direct sky light is normally intercepted by baffles,
and these are shown in the sectional view of the ST optics in Figure
\ref{fig3}. The need for, and the extent of, baffling usually restrict the
size of the field which can receive light from the whole primary
(unvignetted field).

For ground based, multi-purpose telescopes, Ritchey--Cr\'etien
systems have the disadvantage that coma is increased at all foci other
than that for which the system was optimised. Although a Cassegrain
RC telescope

The geometrical bias of the young Maclaurin was not an
exception in early eighteenth-century British mathematics. Newton,
Halley and Simson were deeply concerned with the restoration of the
works of Greek geometers: in particular, the restoration of
Euclid's {\it Porisms\/} became a dominant programme, which they
inherited from their sixteenth-century predecessors like Commandin\'o.
They were motivated by the genuine belief that the geometrical analysis
of the ancients was superior to the modern techniques of the calculus.
The ancients were thought to have concealed their {\it resolutio\/}
and to have presented only the synthetic demonstrations. As
Newton wrote:
\begin{quote}
Indeed their method is more elegant by far than the Cartesian
one. For he achieved the results by an algebraical calculus which,
when transposed into words (following the practice of the Ancients in
their writings), would prove to be so tedious and entangled as to
provoke nausea, nor might it be understood. But they accomplished it
by certain simple propositions, judging that nothing written in a
different style was worthy to be read, and in consequence concealing
the analysis by which they found their constructions.
\begin{flushright}
{\it (Newton (1967--81), IV, p. 277)}
\end{flushright}
\end{quote}
The myth of the power of Greek geometry was part of a more
general attitude towards ancient science. It is known that Newton was
convinced that the ancients had discovered the general laws of the
motions of the planets. This programme, rather than the Newtonian
mechanics and astronomy, motivated Robert Simson, the young Colin
Maclaurin, and later Matthew Stewart.

\begin{theorem}
 Let $\phi$ be integrable (that is,
$\phi \in L_1$) and let $\phi$ vanish
outside a compact subset $ K $ of $\Omega$. We have the following
assertions:
\begin{enumerate}
 \renewcommand{\theenumi}{{\rm(\alph{enumi})}}
 \item
The support of $ \phi = \phi \ast h_\varepsilon $ is contained in
$ K_\varepsilon = \{ x : d(x, K) \leq \varepsilon \},$
 and is again compact.

 \item If $\varepsilon < d(K, {\rm C} \Omega)$,
 then $\phi_\varepsilon \in {\cal D\/}(\Omega)$.
\end{enumerate}
\end{theorem}
\begin{proof}
By Theorem 1.1, Corollary 1.1, there exists
a countable, locally finite refinement
$\{ U_j, j \in {\blackboard N} \}$ with
$U_j = B(x_j, a_j)$ and
$ \overline{U}_j \subset \Omega_{i(j)} $.
We take the functions (4),
$ \phi_{x_j, a_j}(x)$,
because of local finiteness,
there are only finitely many members differing from zero on $B(x, \rho)$.
Hence $\psi \in C^\infty(\Omega)$, and by the covering property together
with (5),
$\psi(x) > 0 $ for all $ x \in \Omega$.
 We define
$\alpha_j(x) = \phi_{x_j,a_j}(x)/\psi(x) $
and see that (a) and (b) are satisfied.
\end{proof}

The central dilatation, which is a geometrical multiplication from
$J$, is carried out by the instantaneous factor
\begin{displaymath}
\lambda = \frac{\overline{JA_j}}{\overline{JA}}
      = \frac{\overline{JB_j}}{\overline{JB}}
      = \frac{\overline{JC_j}}{\overline{JC}},
\end{displaymath}
which is the same factor for all points of the plane. The central
dilatation, thus defined, is followed by a rotation about $J$
over the instant angle
\begin{displaymath}
\theta = A J A_j
      = B J B_j
      = C J C_j
\end{displaymath}
which is also the same for all points of the plane. A spiral
similarity, so carried out, transforms the vertices of
$ \Delta A B C$
into the vertices of a similar triangle, which is
$  \Delta A_j B_j C_j$
in this case.
This proves the theorem (see Fig.~\ref{fig13}). We may use the theorem to
determine the acceleration of any point, if the accelerations of two
points are given. This then by-passes the need for the location of
the acceleration centre.

\subsection{Shower parameters}

In the following, the current state of knowledge concerning the
{\it average\/} parameters of showers will be presented. It should be
understood that they describe {\it individual\/} showers rather
poorly; in particular, they are of limited usefulness for
simulating hadronic showers. Few efforts have been made to describe
the important shower {\it fluctuations\/} in more than qualitative
terms. For the lateral shower shape, even average formulae are not
available. If showers are recognized and used in some specific physics
context, the relevant parameters are, of course, those that describe
the physical track, the jet, and the energy density in part of the
geometrical space. Shower parameters as given in this section may be
useful for interpolating between measured points, or for some aspects
of simulation.

\subsubsection{Longitudinal shower shape}

\paragraph{Electromagnetic showers} Many authors have based
an average electromagnetic shower description on analytical calculations
published by Rossi (1965). His so-called approximation B (constant
energy loss $\varepsilon$ per radiation length $X_0$, valid
at all energies of the high-energy approximation for radiation and
pair production) results in formulae for the shower maximum (the depth
at which ${\rm d}E/{\rm d}x$ in the shower reaches a maximum),
the shower energy median (the depth at which half the shower's energy
has been dissipated), and the shower attenuation (the exponential
damping slope in the shower tail). For incident electrons of energy
$E$ (incident photons result in a slightly stretched shower),
the parameters are given as
\begin{equation} % we are treating this two-line alignment as one array
\begin{array}[b]{rl}    % [b] means its centerline will be the bottom row
 \hbox{Shower maximum at:} & X_0 \left[\, \ln (E/\varepsilon) - 1 \right] \\
 \hbox{Shower energy median at:} & X_0 \left[\, \ln (E/\varepsilon) +
                0.4\right]
\end{array}
\end{equation}

In these formulae, $\varepsilon$  is the {\it critical energy\/},
which is defined as the energy loss by collisions per $X_0$.
The approximate crossover point, at which an electron loses
energy equally by bremsstrahlung and by ionization is at $ E = \varepsilon$.
An approximate value for $\varepsilon$ (good for high
Z materials) is given by $ \varepsilon = 5501 Z$ [MeV],
numerical values are given by Iwata (1980).

\section*{Appendix to Chapter 3}

Calculations of the field were made for all the modes in the cold
homogeneous magnetoplasma by Al'pert {\it et al}. (1983). The results
were published after the first edition of this book was printed (see
also Al'pert \& Moiseyev, 1980; and Al'pert, 1983). They cover a full
frequency range for a proton plasma (Fig. 8.27) and for a plasma with
three species of positive ions. In addition to the features of the
field of the whistler mode given above, the results of this study are
briefly described below\ldots

\section*{Acknowledgements}

The preparation of this publication was supported in part by grants from
the UNDP/World Bank/World Health Organization Special Programme
for Research and Training in Tropical Diseases, and from CNPq and
FINEP, Brazil.

\begin{thereferences}{widest citation in source list}

\bibitem{key1}
Adams, M. A. \& Attiwill, P. M. (1986a). Nutrient
cycling and nitrogen mineralization in eucalypt forests of
south-eastern Australia.\\ I. Nutrient cycling and nitrogen turnover.
{\it Plant and Soil}, {\bf 92}, 319--39.
\bibitem{key2}
Adams, M. A. \& Attiwill, P. M. (1986b). Nutrient
cycling and nitrogen mineralization in eucalypt forests of
south-eastern Australia. II. Indices of nitrogen mineralization. {\it Plant
and Soil}, {\bf 92}, 341--62.
\bibitem{key3}
Alexander, I. J. (1983). The significance of ectomycorrhizas
in the nitrogen cycle. In {\it Nitrogen as an Ecological Factor,
 22nd Symposium of the British Ecological Society}, ed.
J. A. Lee, S. McNeill and I. H. Rorison,
pp. 69--93. Oxford: Blackwell Scientific Publications.
\bibitem{key4}
Allen, O. N. \& Allen, E. K. (1981). {\it The
Leguminosae -- A Source Book of Characteristics, Uses
and Nodulation}. London: Macmillan.
\bibitem{key5}
Anderson, J. M., Leonard, M. A., Ineson, P. \& Huish,
S. (1985). Faunal biomass: a key component of a general model
of nitrogen mineralization.  {\it Soil Biology and Biochemistry},
{\bf 17}, 735--7.
\bibitem{key6}
Anderson, O. E., Boswell, F. C. \& Harrison, R. M.
(1971). Variation in low temperature adaptability of nitrifiers in
acid soils. {\it Soil Science Society of America Proceedings},
{\bf 35}, 68--71.
\bibitem{key7}
Andrews, M. (1986). The partitioning of nitrate assimilation
between root and shoot of higher plants. {\it Plant, Cell and
Environment}, {\bf 9}, 511--19.
\bibitem{key8}
Anon. (1978). {\it Nitrates:  An Environmental
Assessment}. Washington DC: National Academy of Sciences.
\bibitem{key88}
Anon. (1983). {\it The Nitrogen Cycle of the United Kingdom. A
Study Group Report}. London: The Royal Society.
\bibitem{key9}
Anon. (1986). {\it Report of the Acid Rain Enquiry}.
Edinburgh: The Scottish Wildlife Trust.
\bibitem{key10}
Armstrong, W. \& Wright, E. J. (1975). Radial oxygen loss from
roots: the theoretical basis for the manipulation of flux
data obtained by the cylindrical platinum electrode technique.
{\it Physiologia Plantarum}, {\bf 35}, 21--6.
\bibitem{key11}
Ashgar, M. \& Kanehiro, Y. (1976). Effect of sugarcane trash
and apple residue incorporation on soil nitrogen, pH and redox
potential. {\it Plant and Soil}, {\bf 44}, 209--18.
\bibitem{key12}
Atkinson, M. J. \& Smith, S. V. (1983). C:N:P ratios of
benthic plants. {\it Limnology and
Oceanography}, {\bf 28}, 568--74.
\bibitem{key13}
Aumen, N. G., Bottomley, P. J. \& Gregory, S. V.
(1985). Nitrogen dynamics in stream wood samples incubated with
[${}^{14}$C] lignocellulose and potassium [${}^{15}$N] nitrate.
{\it Applied and Environmental Microbiology}, {\bf 49}, 1119--23.
\bibitem{key14}
Bacon, P. E., McGarity, J. W., Hoult, E. H. \&
Alter, D. (1986). Soil mineral nitrogen concentration within cycles
of flood irrigation: Effect of rice stubble and fertilization
management. {\it Soil Biology and Biochemistry}, {\bf 18}, 173--8.
\bibitem{key15}
Bavel, C. H. M. van \& Baker, J. M. (1985). Water
transfer by plants from wet to dry soil. {\it Naturwissenschaften},
{\bf 72}, 606--7.
\bibitem{key16}
Bell, D. T., Hopkins, A. J. M. \& Pate, J. S.
(1982). Fire in the Kwongan. In {\it Kwongan -- Plant Life
of the Sand Plain \/} ed. J. S. Pate \& J. S.
Beard, pp. 178--204. Nedlands: University of Western Australia
Press.
\bibitem{key17}
Bingham, D. R., Lin, C-H. \& Hoag, S. (1984). Nitrogen cycle
and algal growth modeling. {\it Journal of the Water Pollution Control
Federation}, {\bf 56}, 1118--22.
\bibitem{key18}
Bishop, P. E., Premakumar, R., Dean, D. R., Jacobson,
M. R., Chisnell, J. R., Rizzo, T. M. \& Kopczynski, J.
(1986). Nitrogen fixation by {\it Azobacter vinelandii\/} strains
having deletions in structural genes for nitrogenase. {\it Science},
{\bf 232}, 92--4.
\bibitem{key19}
Bliss, L. C., Heal, O. W. \& Moore, J. J. (eds.)
(1981). {\it Tundra Ecosystems: A Comparative
Analysis}. The International Biological Programme 25. Cambridge:
Cambridge University Press.
\bibitem{key20}
Bloom, A. J., Chaplin, F. S. \& Mooney, H. A.
(1985). Resource limitation in plants -- an economic
analogy. {\it Annual Review of Ecology and Systematics}, {\bf 16},
363--92.
\bibitem{key21}
Bolin, B. \& Cook, R. B. (eds.) (1983). {\it The Major
Biogeochemical Cycles and their Interactions}. Scope 21.
Chichester: John Wiley \& Sons.
\bibitem{key22}
Boudot, J. P. \& Chone, Th. (1985). Internal nitrogen
cycling in two humic-rich acidic soils. {\it Soil Biology and
Biochemistry}, {\bf 17}, 135--42.
\bibitem{key23}
Bowen, G. D. \& Smith, S. E. (1981). Effect of
mycorrhizas on nitrogen uptake by plants. In {\it Terrestrial
Nitrogen Cycles}, Ecological Bulletin 33, ed. F. E.
Clark \& T. Rosswall, pp. 237--48. Stockholm: Swedish
Natural Science Research Council.
\bibitem{key24}
Briggs, G. C. (1975). The behaviour of the nitrification
inhibitor `N-Serve' in broadcast and incorporated
applications to soil. {\it Journal of the Science of Food and
Agriculture}, {\bf 26}, 1083--92.
\bibitem{key25}
Brown, R. H. (1978). A difference in N use efficiency in
$ {\rm C}_3$ and$ {\rm  C}_4$ plants and its implications in adaptation
and evolution. {\it Crop Science}, {\bf 18}, 93--8.
\bibitem{key26}
Bunnell, F. L., Maclean, S. F. Jr \& Brown, J.
(1975). Barrow, Alaska, USA. In {\it Structure and Functions of
Tundra Ecosystems}, Ecological Bulletin 20, ed.  T. Rosswall
\& O. W. Heal, pp. 73--124. Stockholm: Swedish Natural
Science Research Council.
\bibitem{key27}
Burns, R. G. (ed.) (1978). {\it Soil Enzymes}. London:
Academic Press.
\bibitem{key28}
Burns, T. A. Jr, Bishop, P. E. \& Israel, D. W.
(1981). Enhanced nodulation of leguminous plant roots by mixed
cultures of {\it Azobacter vinelandii\/} and {\it Rhizobium. Plant
and Soil}, {\bf 62}, 399--412.
\bibitem{key29}
Burris, R. H. (1976). Nitrogen fixation by blue--green
algae of the Lizard Island area of the Great Barrier Reef.
{\it Australian Journal of Plant Physiology}, {\bf 3}, 41--51.
\bibitem{key30}
Burris, R. H. (1983). Uptake and assimilation of
${}^{15}{\rm NH}^+_4$ by a variety of corals. {\it Marine
Biology}, {\bf 75}, 151--5.
\bibitem{key31}
Calvert, J. G., Lazarus, A., Kok, G. K., Heikes,
B. G., Walega, J. G., Lind, J.  \& Cantrell, C. A.
(1985). Chemical mechanisms of acid generation in the troposphere.
{\it Nature}, {\bf 317}, 27--35.
\bibitem{key32}
Campbell, R. (1985). {\it Plant Microbiology}. London:
Edward Arnold.
\bibitem{key33}
Canfield, D. E. \& Greem, W. J. (1985). The cycling of
nutrients in a closed-basin Antarctic lake: Lake Vanda.
{\it Biogeochemistry}, {\bf 1}, 233--56.
\bibitem{key34}
Canuto, V. M., Levine, J. S., Augustsson, T. R.
\& Imhoff, C. L. (1982). UV radiation from the young sun and
oxygen and ozone levels in the prebiological atmosphere. {\it Nature},
{\bf 296}, 816--20.
\bibitem{key35}
Carpenter, E. J. (1983). Nitrogen fixation by marine
{\it Oscillatoria (Trichodesium)\/} in the world's oceans. In
{\it Nitrogen in the Marine Environment}, ed. E. J.
Carpenter \& D. G. Capone, pp. 65--103. New York:
Academic Press.
\end{thereferences}

\clearpage
\author[H. E. Bishop]{H. E. BISHOP\\Affiliation}
\chapter{Auger electron spectroscopy}

\clearpage
\author[A. B. Christie]{A. B. CHRISTIE\\Affiliation}
\chapter{X-ray photoelectron spectroscopy}

\clearpage
\author[J. C. Vickerman]{J. C. VICKERMAN\\Affiliation}
\chapter{Static secondary ion mass spectroscopy}

\begin{theauthorindex}
\item Abrams, S. I 279
\item Ackerson, K. L. II 122, 229, 239
\item Adachi, S. I 217, 271
\item Aikyo, K. I 216, 238, 241, 243--5, 271 II 151, 224
\item Akasofu, S. I. II 242
\item Akhiezer, A. I. I 28, 37, 43, 61, 94, 101, 105, 109, 111,
      118, 135, 207, 210, 271
\item Akhiezer, I. A. I 271
\item Alexander, J. K. II 100, 214, 224
\item Alexeff, I. I 204, 279
\item Aliev, Yu. M. I 193--5, 199, 271 II 224
\item Allen, E. M. I 271 II 224
\item Al'pert, Ya. L. I 22, 30, 37, 40, 46, 68, 71, 73, 76, 77,
      79, 141, 143, 216, 217, 219, 221, 225, 227, 231, 233, 246,
      249--51, 260, 264--7, 269, 271, 272 II 3, 4, 15, 38, 63, 65,
      67--9, 75, 79, 80, 82, 89, 106--8, 160, 190, 224
\item Al'tshul', L. M. I 133, 135, 272
\item Anderson, R. R. II 128, 129, 132, 177, 196, 200, 218, 221,
      225, 230, 231, 241
\item Andreev, N. E. I 200--4, 272
\item Angerami, J. J. I 15, 272  II 151, 152, 160, 161, 225, 241, 242
\item Appleton, E. V. I 49, 272
\item Ashour-Abdalla, M. I 78, 272 II 163, 189, 225
\item Astrelin, V. T. II 6, 26, 225
\item Aubry, M. P. I 217, 238, 272, 273 II 226
\indexspace
\item Bahnsen, A. II 228
\item Bailey, V. A. I 135, 273
\item Baker, B. II 225
\item Barfield J. N. II 225
\item Barfield, T. A. I 109, 282
\item Barkhausen, H. II 160, 225
\item Barrett, P. J. II 6, 28, 225
\item Barrington, R. E. II 90, 111, 112, 138, 149, 150, 182, 225,
      231, 235, 240
\item Bashilov, I. P. II 224
\item Baumback, M. M. II 234
\item Beghin, C. II 190, 225, 227, 231, 241
\item Belikovich, V. V. I 160, 161, 168, 169, 273
\item Bell, T. F. I 213, 273 II 232
\item Belrose, J. S. II 138, 149, 150, 225, 240
\item Benediktov, E. A. I 273
\item Benioff, H. II 103, 225
\item Berezin, Yu. A. I 135, 210, 211, 273
\item Bernstein, I. B. I 102, 133, 135, 273
\item Berstein, W. II 191, 225
\item Binsack, J. H. I 280 II 240
\item Biondi, M. A. I 163, 273, 280
\item Bitoun, J. I 217, 238, 272, 273 II 186, 226
\item Blair, W. E. II 231
\item Bloch, J. J. II 227
\item Blood, D. W. I 169, 172, 281
\item Bloom, M. H. II 234
\item Bogashckenko, I. A. II 6, 25, 28, 29, 225, 226
\item Bohm, D. I 108, 273
\item Booker, H. G. I 27, 47 171, 172, 174, 216, 217, 247, 273 II 226
\item Borisov, N. D. I 167, 273
\item Bossen, M. II 110, 217, 226
\item Bourdeau, R. E. II 5, 226
\item Bowen, P. J. II 5, 23, 34, 226
\item Bowhill, S. A. I 167, 273
\item Boyd, R. L. II 226
\item Brace, L. H. I 22, 274
\item Bremmer, H. I 216, 274
\item Brice, N. M. I 36, 277, 281 II 90, 111--13, 138, 163, 226, 230,
      232, 233, 235, 240
\item Bridge, H. S. I 280 II 240
\item Brinton, H. C. I 281 II 237
\item Brown, L. W. I 23, 274 II 208, 212, 226, 236
\item Brown, P. E. II 236
\item Brundin, C. L. II 4, 226
\item Buchel'nikova, N. S. II 225, 230
\item Bud'ko, N. I. II 47--52, 55--57, 88, 90, 91, 226
\item Budden, K. G. I 47, 66, 217, 269, 274
\item Bullough, K. II 167, 226, 232, 234, 241
\item Burns, T. B. II 117, 118, 230
\item Burtis, W. J. II 169--172, 226
\item Buchel'nikova, N. S. II 225, 230
\item Bud'ko, N. I. II 47--52, 55--57, 88, 90, 91, 226
\item Budden, K. G. I 47, 66, 217, 269, 274
\item Bullough, K. II 167, 226, 232, 234, 241
\item Burns, T. B. II 117, 118, 230
\item Burtis, W. J. II 169--172, 226
\end{theauthorindex}

\begin{thesubjectindex}
\item absorption coefficients I 81, 168, 180
\item absorption of particles II 12
\item accommodation of particles, partial and total II 11, 12
\item accumulation of particles near body II 63
\item Aerobee rocket II 149
\item AKR, {\it see\/} auroral kilometric radiation
\item Alfv\'en refractive index I 8, 53, 63
\item Alfv\'en velocity I 8, 43, 63, 203
\item Alv\'en wave I 53, 58, 63, 64, 93, 94, 101, 102, 123, 124, 127,
      203, 222, 237 II 104
\item Alouette satellites II 97, 149, 189
\item Alouette 1 II 90, 91, 110, 138, 151, 179
\item Alouette 2 I 23 II 112, 147, 148, 179--83, 193
\item ambipolar diffusion coefficient I 154
\item amplitude-modulation index I 186
\item angular dependence of electric field near body II 55, 56
\item angular dependence of particles near body II 19--25, 30, 33--41,
      47--54, 57--61
\item angular dependence of temperature near body II 16, 17
\item anisotropic distribution of electron velocities II 178
\item anisotropic electron distribution II 130
\item anisotropic instability I 110ff
\item anisotropic Maxwellian distribution I 110, 117, 126
\item anisotropic pitch angle distribution II 100
\item anisotropic temperature distribution I 106, 107 II 100, 106
\item anistropic velocity distribution I 106, 110ff, 116ff, 126 II 106
\item anomalous absorption I 167--9
\item anomalous Doppler effect I 111
\item Antarctica, {\it see\/} Siple, Eights station
\item Apollo model II 43, 44
\item Appleton--Hartree formula I 48, 49
\item Appleton--Lassen formula I 49
\item Ariel 1 satellite II 5, 17, 23, 34, 35
\item Ariel 3 satellite II 148, 167
\item Ariel 4 satellite II 167
\item artificial sporadic layer I 163, 164, 167 II 97
\item artificially created emissions II 194, 195
\item artificially stimulated emissions (ASE) II 162ff
\item ASE, {\it see\/} artificially stimulated emissions, stimulated emission
\item ATS I satellite II 109, 110
\item ATS 5 satellite I 21
\item attenuation coefficient I 70ff, 82ff, 147, 184
\item attenuation factor I 28, 38, 43, 45, 70, 71
\item aurorae II 213, 214
\item auroral activity II 136, 176
\item auroral hiss (AH) II 137, 142ff, 146, 147
\item auroral kilmetric radiation (AKR) II 208ff, 222
\item auroral oval II 124, 144
\item auroral region I 21
\indexspace
\item backscatter I 163, 172, 191
\item bank structure of chorus II 174
\item band structure of TNCR II 207
\item beam deformation I 190
\item beam instability I 31, 106, 107ff, 118ff II 100, 101, 178, 215
\item beam instability of longitudinal waves I 111ff
\item beam instability of transverse (e.m.) waves I 115ff, 123ff
\item beam of particles I 31, 107ff
\item beam of electromagneic waves I 153, 184, 188
\item Bernstein modes I 102 II 185, 194
\item Bessel function (imaginary argument) I 102, 196, II 20, 61, 77
\item Bessel function (real argument) I 194, 261, 262 II 77
\item bi-Maxwellian function I 117
\item Boltzmann distribution function I 29, 30, 214
\item boundary conditions at body's surface I 7 II 10 (ch. 10)
\item boundary layer near body II 66
\item Bessel function (imaginary argument) I 102, 196, II 20, 61, 77
\item Bessel function (real argument) I 194, 261, 262 II 77
\item bi-Maxwellian function I 117
\item Boltzmann distribution function I 29, 30, 214
\item boundary layer near body II 66
\end{thesubjectindex}

\end{document}
%%%%%%%%%%%%%%%%%%%%%% end of cup6b.tex %%%%%%%%%%%%%%%%%%%%%%
