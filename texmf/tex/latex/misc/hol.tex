% $Id$
%% Higher-Order logic a la Back and von Wright (HOL style),
%% with calculational proofs etc.

% requires calculation.tex

% Change the style of brackets from calculation.tex
\renewcommand{\LEFTREASONBRACKET}{\{}
\renewcommand{\RIGHTREASONBRACKET}{\}}
\renewcommand{\BYPHRASE}{\relax}

\newcommand{\BIGDOT}{\mbox{$\bullet$}}

\newcommand{\SUBPROOF}{\+\+ \kill \BIGDOT}
\newcommand{\ASSUMPTIONSUBPROOF}[1]{\+\+ \kill \BIGDOT \>\>[#1] \\}
\newcommand{\ENDSUBPROOF}{\-\- \kill}
\newcommand{\CONCLUSIONANDLAST}[1]{\mbox{$\cdot$} \LASTFORMULA{#1}}
\newcommand{\CONCLUSIONOFSUBPROOF}[1]{\CONCLUSIONANDLAST{#1} \\}

\newcommand{\EQUIVREASON}[1]{\REASON{\equiv}{#1}}
\newcommand{\TRUTH}{\mbox{T}}
\newcommand{\FALSITY}{\mbox{F}}

%% The HOLcalculation has no assumptions, while the HOLproof form does
\newenvironment{HOLcalculation}{
\samepage
\begin{tabbing}
mm \= n \= mm \= n \= mm \= n \= mm \= n \= mm \= n \= \kill
\mbox{$\vdash$}
}{
\end{tabbing}
}

\newenvironment{HOLproof}[1]{
\samepage
\begin{tabbing}
mm \= n \= mm \= n \= mm \= n \= mm \= n \= mm \= n \= \kill
\mbox{$#1$} \\
\mbox{$\vdash$}
}{
\end{tabbing}
}

%% Typesetting of program calculations

\newcommand{\ASSIGN}{\mbox{ := }}
\newcommand{\IS}{\ensuremath{\widehat{=}}}
\newcommand{\VAR}{\mbox{var~}}
\newcommand{\SET}{\mbox{set}}
