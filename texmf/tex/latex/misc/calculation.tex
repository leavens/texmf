% $Id$
%
% Calculational proofs, a la Dijkstra, Gries, Back and von Wright, etc.
% See: David Gries, "Teaching Calculation and Discrimination:
%      A More Effective Curriculum", CACM 34(3):44-55, Mar. 1991.
% and
%      Ralph-Johan Back and Joakim von Wright,
%      "Refinement Calculus: A Systematic Introduction",
%      Springer-Verlag, 1998.
%
% AUTHOR: Gary T. Leavens

% EXAMPLES
%
% A simple example of the use of these macros is as follows.
%
% \begin{calculation}
% \FORMULA{\vec{RD}}
% \EQREASON{definition of $\vec{RD}$}
% \FORMULA{(RD_1, \ldots, RD_j, \ldots, RD_{12})}
% \REASON{\sqsubseteq}{$F_j$ is monotonic}
% \FORMULA{(RD_1, \ldots, F_j(\vec{RD}), \ldots, RD_{12})}
% \EQREASON{definition of $\vec{RD'}$}
% \LASTFORMULA{\vec{RD'}}
% \end{calculation}
%
% The following is a slightly more advanced example.
%
% \begin{calculation}
% \FORMULA{F(\vec{RD'}) \sqsubseteq \FOCUS{F^n(\vec{\emptyset})}}
% \EQREASON{assumption $F^n(\vec{\emptyset}) = F^{n+1}(\vec{\emptyset})$}
% \FORMULA{F(\vec{RD'}) \sqsubseteq \FOCUS{F^{n+1}(\vec{\emptyset})}}
% \EQREASON{definition of iteration of a function}
% \FORMULA{F(\vec{RD'}) \sqsubseteq F(F^n(\vec{\emptyset}))}
% \REASON{\impliedby}{monotonicity of $F$}
% \FORMULA{\vec{RD'} \sqsubseteq F^n(\vec{\emptyset})}
% \REASON{\impliedby}{transitivity}
% \FORMULA{\vec{RD'} \sqsubseteq F(\vec{RD})
%         \wedge F(\vec{RD}) \sqsubseteq F^n(\vec{\emptyset})}
% \EQREASON{assumption $F(\vec{RD}) \sqsubseteq F^n(\vec{\emptyset})$}
% \FORMULA{\vec{RD'} \sqsubseteq F(\vec{RD})}
% \REASON{\impliedby}{Lemma~7}
% \QEDFORMULA{\vec{RD} \sqsubseteq F(\vec{RD})}
% \end{calculation}
%
% An much more sophisticated example, with subdeductions a la Back and
% von Wright is the following.
%
% \begin{calculation}
% \FORMULA{\alpha(\{tr: (z, \ell') \mid tr \in \gamma(RD_k)\}) }
% \EQREASON{definition of $\alpha$}
% \FORMULA{\{(x, SRD(tr')(x))
%              \mid x \in \textrm{dom}(tr')
%                   \wedge tr' \in \{tr: (z, \ell') \mid tr \in \gamma(RD_k)\} \} }
% \EQREASON{set theory
%          ($tr'$ ranges over $tr: (z, \ell')$ for $tr \in \gamma(RD_k)$)}
% \FORMULA{\{(x, SRD(tr: (z, \ell'))(x))
%              \mid x \in \textrm{dom}(tr: (z, \ell')),
%                    \FOCUS{tr \in \gamma(RD_k)} \} }
% \EQREASON{definition of $\gamma$}
% \FORMULA{\{(x, SRD(tr: (z, \ell'))(x))
%              \mid x \in \textrm{dom}(tr: (z, \ell')),
%                (\forall y \in \textrm{dom}(tr) :: (y, SRD(tr)(y)) \in RD_k) \}}
% \EQREASON{set theory}
% \FORMULA{
%     \begin{array}{rl}
%         \{(x, SRD(tr: (z, \ell'))(x)) &
%              \mid x \neq z, x \in \textrm{dom}(tr: (z, \ell')), \\
%              & ~ (\forall y \in \textrm{dom}(tr) :: (y, SRD(tr)(y)) \in RD_k) \} \\
%         \cup \:
%         \{(x, SRD(tr: (z, \ell'))(x)) &
%              \mid x = z, x \in \textrm{dom}(tr: (z, \ell')), \\
%              & ~ (\forall y \in \textrm{dom}(tr) :: (y, SRD(tr)(y)) \in RD_k) \}
%     \end{array}
%         }
% \EQREASON{focusing on the second set}
% \SUBDEDUCTION{
%      \begin{array}{rl}
%         \{(x, SRD(tr: (z, \ell'))(x)) &
%              \mid x = z, x \in \textrm{dom}(tr: (z, \ell')), \\
%              & ~ (\forall y \in \textrm{dom}(tr) :: (y, SRD(tr)(y)) \in RD_k) \}
%      \end{array}
%        }
%   \EQREASON{set theory}
%   \FORMULA{\{(z, \FOCUS{SRD(tr: (z, \ell'))(z)}) 
%              \mid z \in \textrm{dom}(tr: (z, \ell')),
%                (\forall y \in \textrm{dom}(tr) :: (y, SRD(tr)(y)) \in RD_k) \} }
%   \EQREASON{definition of $SRD$}
%   \FORMULA{\{(z, \ell') 
%              \mid \FOCUS{z \in \textrm{dom}(tr: (z, \ell'))},
%                (\forall y \in \textrm{dom}(tr) :: (y, SRD(tr)(y)) \in RD_k)
%              \} }
%   \EQREASON{definition of $\textrm{dom}$}
%   \FORMULA{\{(z, \ell') 
%              \mid (\forall y \in \textrm{dom}(tr) :: (y, SRD(tr)(y)) \in RD_k) \} }
%   \EQREASON{definition of $\gamma$}
%   \FORMULA{\{(z, \ell') \mid tr \in \gamma(RD_k) \} }
%   \EQREASON{$tr$ can be constructed by writing the pairs of
%              $\gamma(RD_k)$ in some order}
%   \FORMULA{\{(z, \ell') \} }
% \CONCLUDE{
%    \begin{array}{l}
%     \begin{array}{rl}
%         \{(x, SRD(tr: (z, \ell'))(x)) &
%              \mid x \neq z, x \in \textrm{dom}(tr: (z, \ell')), \\
%              & ~ (\forall y \in \textrm{dom}(tr) :: (y, SRD(tr)(y)) \in RD_k) \}
%     \end{array} \\
%     \cup \: \{(z, \ell')\}
%    \end{array}
%     }
% \EQREASON{focusing on the first set}
% \SUBDEDUCTION{
%     \begin{array}{rl}
%         \{(x, SRD(tr: (z, \ell'))(x)) &
%              \mid x \neq z, x \in \textrm{dom}(tr: (z, \ell')), \\
%              & ~ (\forall y \in \textrm{dom}(tr) :: (y, SRD(tr)(y)) \in RD_k) \}
%      \end{array}
%        }
%   \EQREASON{definition of $SRD$}
%   \FORMULA{\{(x, SRD(tr)(x))
%              \mid x \neq z, x \in \textrm{dom}(tr), 
%               (\forall y \in \textrm{dom}(tr) :: (y, SRD(tr)(y)) \in RD_k) \}}
%   \EQREASON{set theory and the range of $SRD$}
%   \FORMULA{\begin{array}{l}
%              \{(x, SRD(tr)(x))
%                       \mid x \in \textrm{dom}(tr), 
%               (\forall y \in \textrm{dom}(tr) :: (y, SRD(tr)(y)) \in RD_k) \} \\
%            \setminus \: \{(z, \ell) \mid \ell \in \Lab\}
%            \end{array}
%           }
%   \EQREASON{Lemma \ref{lemma:simplifyToRD}}
%   \FORMULA{RD_k \setminus \{(z, \ell) \mid \ell \in \Lab\}}
% \LASTCONCLUDE{(RD_k \setminus \{(z, \ell) \mid \ell \in \Lab\})
%            \cup \{(z, \ell')\}}
% \end{calculation}


%%%%%%% The macros themselves follow %%%%%%%%%%%%%%%%%%%%%%%%%%%%

%%% You may want to define the following with \renewcommand...

% Brackets around normal and fixed width reasons
\newcommand{\LEFTREASONBRACKET}{\langle}
\newcommand{\RIGHTREASONBRACKET}{\rangle}
% Brackets for long reasons
\newcommand{\LONGLEFTREASONBRACKET}{\left\LEFTREASONBRACKET}
\newcommand{\LONGRIGHTREASONBRACKET}{\right\RIGHTREASONBRACKET}
% width of long and fixed width reasons
\newcommand{\LONGREASONWIDTH}{5.4in}
\newcommand{\FIXEDWIDTHREASONWIDTH}{\LONGREASONWIDTH}
% The phrase that starts each reason
\newcommand{\BYPHRASE}{by~}
% Provability symbol
\newcommand{\PROVABILITYSYMBOL}{\vdash}
% Has type symbol
\newcommand{\CALCULATIONHASTYPE}{:}
% Subdeduction symbols a la Back and von Wright
\newcommand{\SUBDEDUCTIONMARKER}{\bullet}
\newcommand{\CONCLUDEMARKER}{\cdot}

% The macros below assume that you have defined \QED,
% e.g., as \newcommand{\QED}{q.e.d.}


%%% The rest shouldn't usually need to be redefined

% The main environment
\newenvironment{calculation}{
\samepage
\begin{tabbing}
mm \= n \= mm \= n \= mm \= n \= mm \= n \= mm \= n \= mm \= n \= \kill
}{
\end{tabbing}
}

% Use for the last formula (to avoid extra vertical space)
\newcommand{\LASTFORMULA}[1]{\> \ensuremath{#1}}
% Use for the last formula in a proof that concluses the proof
\newcommand{\QEDATEND}{~~\QED}
\newcommand{\QEDFORMULA}[1]{\LASTFORMULA{#1\QEDATEND}}

% All other formulas in a calculation
\newcommand{\FORMULA}[1]{\LASTFORMULA{#1} \\}

% Generic reason, with connective and the justification
\newcommand{\REASON}[2]{%
    \ensuremath{#1}\> \>
       \mbox{\ensuremath{\LEFTREASONBRACKET}{\BYPHRASE}#2\ensuremath{\RIGHTREASONBRACKET}} \\}

% Reasons with longer justifications, with the right bracket at the end
\newcommand{\FIXEDWIDTHREASON}[2]{%
     \ensuremath{#1}\> \>
       \ensuremath{\LEFTREASONBRACKET}{\BYPHRASE}%
       \parbox[t]{\FIXEDWIDTHREASONWIDTH}{#2\ensuremath{\RIGHTREASONBRACKET}}
       \\
     }

% Reasons with longer justifications, with big brackets around the reason
\newcommand{\LONGREASON}[2]{%
    \ensuremath{#1}\> \>
      \ensuremath{\LONGLEFTREASONBRACKET
                  \parbox[c]{\LONGREASONWIDTH}{{\BYPHRASE}#2}
                  \LONGRIGHTREASONBRACKET
                 }
      \\
    }

% Specialization of reasons to equality
\newcommand{\EQREASON}[1]{\REASON{=}{#1}}
\newcommand{\EQLONGREASON}[1]{\LONGREASON{=}{#1}}
\newcommand{\EQFIXEDWIDTHREASON}[1]{\FIXEDWIDTHREASON{=}{#1}}

% Use \FOCUS{E} to say that subformula E will be changed in the following step
\newcommand{\FOCUS}[1]{\mbox{``}#1\mbox{''}}


%% The following are for Back and von Wright style calculations.
%% First time users may want to ignore these.

% Use for assumptions in a type-deduction style proof
\newcommand{\GIVEN}[1]{\FORMULA{#1}}
% Use \PROVETHAT to start the body of a type-deduction style proof.
\newcommand{\PROOFLINE}[2]{\ensuremath{#1} \> \ensuremath{#2} \\}
\newcommand{\PROVETHAT}[1]{\PROOFLINE{\PROVABILITYSYMBOL}{#1}}
%% Subdeductions
% Use to start a subdeduction
\newcommand{\SUBDEDUCTION}[1]{\+\+\kill\PROOFLINE{\SUBDEDUCTIONMARKER}{#1}}
% Use to separate one subdeduction from another
\newcommand{\ANDALSO}{\-\-\kill}
% Use to start the next subdeduction after \ANDALSO
\newcommand{\NEXTDEDUCTION}[1]{\PROOFLINE{\SUBDEDUCTIONMARKER}{#1}}
% Use to conclude a subdeduction 
\newcommand{\CONCLUDE}[1]{\-\-\kill\PROOFLINE{\CONCLUDEMARKER}{#1}}
% Use when the last formula is also a conclusion
\newcommand{\LASTCONCLUDE}[1]{\-\-\kill\ensuremath{\CONCLUDEMARKER} \> \ensuremath{#1}}
% Use in a type-deduction proof instead of \REASON, to give the rule name
\newcommand{\TYPINGRULE}[1]{\ensuremath{\CALCULATIONHASTYPE} \> ~~~ #1 \\}

% end of calculation.tex
