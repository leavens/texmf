From guttag@larch.lcs.mit.edu Wed Mar 10 11:29:37 1993
Return-Path: <guttag@larch.lcs.mit.edu>
Date: Tue, 9 Mar 93 09:56:13 EST
From: guttag@larch.lcs.mit.edu (John Guttag)
To: leavens@cs.iastate.edu
Subject: Re:  larch.sty in SliTeX?

Gary,

I have some macros that I think work pretty well, but I use them in
the context of regular LaTeX.

John
-----------------
\documentstyle[fleqn,12pt,/u/guttag/tex/larch]{article}
\font\latexsyms=lasy10
\tolerance=1200
\pretolerance=1000
\hyphenpenalty=600

\nosuperscripts
\nosubscripts

\renewcommand{\topfraction}{.95}
\renewcommand{\bottomfraction}{.95}
\renewcommand{\textfraction}{.05}
\renewcommand{\floatpagefraction}{.70}
\setcounter{topnumber}{4}
\setcounter{bottomnumber}{3}
\setcounter{totalnumber}{5}

\newcommand{\f}[1]{{\tt #1}}
\renewcommand{\fbox}[1]{\,\mbox{\f{#1}}\,}
\renewcommand{\d}[1]{{\it #1\/}}
\newcommand{\dbox}[1]{\mbox{\d{#1}}}

\newcommand{\sfig}{\begin{figure}[tbp]}
\newcommand{\efig}[2]{\caption{#1}\label{#2}\end{figure}}

\newcommand{\verbi}[1]{\begin{verbatim} \input{#1}}
\newcommand{\speci}[1]{\begin{spec} \LARGE\tt \input{#1}}
\newcommand{\form}{\begin{spec} \LARGE\tt}

\mathrename{\forall}{\mathforall}
\mathrename{\exists}{\mathexists}
\mathrename{\in}{\mathin}
\mathrename{\ominus}{\mathominus}
\mathrename{\subset}{\mathsubset}
\mathrename{\subseteq}{\mathsubseteq}
\mathrename{\neq}{\mathneq}
\mathrename{\neg}{\mathneg}
\mathrename{\circ}{\mathcirc}
\mathrename{\sim}{\mathsim}

\newcommand{\inv}{^{-1}}
\mathrename{\inv}{\mathinv}
\newcommand{\lpand}{$\wedge$}
\newcommand{\lpor}{$\vee$}
\newcommand{\lpnot}{$\not$}
\newcommand{\lpimplies}{$Rightarrow$}
\newcommand{\pre}{{\it pre}}
\mathrename{\pre}{\mathpre}
\newcommand{\post}{{\it post}}
\mathrename{\post}{\mathpost}
\newcommand{\reqP}{{\it reqP}}
\mathrename{\reqP}{\mathreqP}
\newcommand{\lpequiv}{\Leftrightarrow}
\mathrename{\lpequiv}{\mathlequiv}
\newcommand{\lequiv}{\Leftrightarrow} % For debugging chapter
\mathrename{\lequiv}{\mathlequiv}     % For debugging chapter
\newcommand{\I}{\cap}
\mathrename{\I}{\mathI}
\newcommand{\U}{\cup}
\mathrename{\U}{\mathU}
\newcommand{\modList}{{\it modList}}
\mathrename{\modList}{\mathmodList}
\newcommand{\ensP}{{\it ensP}}
\mathrename{\ensP}{\mathensP}
\newcommand{\terminates}{{\it terminates}}
\mathrename{\terminates}{\mathterminates}
\newcommand{\modP}{{\it modP}}
\mathrename{\modP}{\mathmodP}
\newcommand{\gd}{{\tt =>}}            % For LM3 guard punctuation
\mathrename{\gd}{\mathgd}
\newcommand{\ra}{{\tt ->}}            % For LCL rightarrow, points to
\mathrename{\ra}{\mathra}
\newcommand{\any}{^\sim}
\mathrename{\any}{\mathany}

\newcommand{\hatt}{$\,^{\wedge}$}
\newcommand{\qb}{{\tt \char'134}}


% Derived from Leslie Lamport's suggestions

    \makeatletter
    \def\partcaption{\@dblarg{\@mycaption\@captype}}
    
    \long\def\@mycaption#1[#2]#3{\par\begingroup
        \@parboxrestore
        \normalsize
        \@makecaption{\csname fnum@#1\endcsname}{\ignorespaces #3}\par
      \endgroup}
    \makeatother

  \newcommand{\multicaption}{}
  \newcounter{multipart}

  \newenvironment{multifigure}[1]%
     {\renewcommand{\multicaption}{#1}
       \setcounter{multipart}{1}
       \begin{figure}[bp]}%
     {\partcaption{\multicaption, part \themultipart}\end{figure}}

  \newcommand{\breakfigure}{
      \ifnum\themultipart=1
         \caption{\multicaption, part \themultipart}\label{\multicaption}
         \else\partcaption{\multicaption, part \themultipart}
         \fi
         \addtocounter{multipart}{1}
       \end{figure}
       \begin{figure}[tpb]}

\textwidth=7.1in
\textheight=24.0cm
\hoffset=-.4in
\voffset=-1in
\parskip=.2in
\leftmargini=12pt
\topsep=0pt
\partopsep=0pt
\itemsep=0pt
\marginparwidth=0.7in
\setlength{\parindent}{0in}
\pagestyle{empty}
\newfont{\slf}{cmr10 scaled\magstep4}
\newfont{\emss}{cmr10 scaled\magstep4}
\newfont{\slbf}{cmb10 scaled\magstep4}
\newfont{\slif}{cmti10 scaled\magstep4}
\newcommand{\rw}{\slbf}
\newcommand{\np}[1]{\newpage\begin{center}{\rw #1}\end{center}\vspace{.25in}}
\newcommand{\ind}{\hspace{25pt}}
\newcommand{\hd}{\slf}
\newenvironment{items}
    {\addtolength{\parindent}{30pt}}%
    {\addtolength{\parindent}{-30pt}}


