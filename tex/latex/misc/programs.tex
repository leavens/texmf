%{programs}
% \makeatletter
% \@ifundefined{myobeycr}{\input{obey}}{\relax}
% \makeatother

\usepackage{xspace}

% one can renew this command to use a different font for program text
\newcommand*{\programFontDeclaration}{\normalfont\ttfamily}

% environment for programs, indented and set in typewriter font
\newcommand{\programCOLspace}{\hspace{1.2em}}
\providecommand{\program}{}	% make sure program is defined, so renew works
\renewenvironment{program}{%
\samepage\programFontDeclaration%
\begin{list}{}{}%
\item\begin{tabbing}%
\programCOLspace \= \programCOLspace \= \programCOLspace \= \programCOLspace \=
\programCOLspace \= \programCOLspace \= \programCOLspace \= \programCOLspace \=
\programCOLspace \= \programCOLspace \= \programCOLspace \= \programCOLspace \=
\kill
}{%
\end{tabbing}\end{list}}

% environment for programs, not-indented, but set in typewriter font
\newenvironment{nonindentedprogram}%
	       {\samepage\programFontDeclaration%
		 \begin{list}{}{\setlength{\leftmargin}{0pt}}%
		 \item\begin{tabbing}%
		   \programCOLspace \= \programCOLspace \= \programCOLspace \= 
		   \programCOLspace \= \programCOLspace \= \programCOLspace \= 
		   \programCOLspace \= \programCOLspace \= \programCOLspace \=
		   \programCOLspace \= \programCOLspace \= \programCOLspace \=
		   \kill
	       }%
	       {\end{tabbing}\end{list}}


% environment for programs, indented and set in typewriter font, and centered
\newenvironment{centeredprogram}%
	       {\nopagebreak\programFontDeclaration%
		 \begin{center}\begin{minipage}{\linewidth}
		 \begin{list}{}{\setlength{\leftmargin}{0pt}}%
		 \item\begin{tabbing}%
		   \programCOLspace \= \programCOLspace \= \programCOLspace \= 
		   \programCOLspace \= \programCOLspace \= \programCOLspace \= 
		   \programCOLspace \= \programCOLspace \= \programCOLspace \=
		   \programCOLspace \= \programCOLspace \= \programCOLspace \=
		   \kill
	       }%
	       {\end{tabbing}\end{list}\end{minipage}\end{center}}

% environment for programs, indented and set in typewriter font, stays on one page
\newenvironment{programNoBreak}[1]{% argument is width, typically use \textwidth or \columnwidth
\programFontDeclaration%
\begin{list}{}{}\begin{minipage}{#1}%
\item\begin{tabbing}%
\programCOLspace \= \programCOLspace \= \programCOLspace \= \programCOLspace \=
\programCOLspace \= \programCOLspace \= \programCOLspace \= \programCOLspace \=
\programCOLspace \= \programCOLspace \= \programCOLspace \= \programCOLspace \=
\kill
}{%
\end{tabbing}\end{minipage}\end{list}}

% environment for programs, indented, set in typewriter font, obeys cr, tab,etc
\newenvironment{program*}{%
\samepage\programFontDeclaration% 
\begin{list}{}{}\item\obeyspaces\obeytabs\obeycr}{%
\end{list}}

% programs with underbars typed as _ instead of \_
% In these environments use \BEGINMATH ... \ENDMATH for math with subscripts.
\newcommand{\BEGINMATH}{\begingroup\catcode`\_=8$}
\newcommand{\ENDMATH}{$\catcode`\_=12\endgroup}
%
% Uprogram is like program
\newenvironment{Uprogram}{%
\samepage\programFontDeclaration%
\begin{list}{}{}\item\begin{tabbing}\catcode`\_=12%
\programCOLspace \= \programCOLspace \= \programCOLspace \= \programCOLspace \=
\programCOLspace \= \programCOLspace \= \programCOLspace \= \programCOLspace \=
\programCOLspace \= \programCOLspace \= \programCOLspace \= \programCOLspace \=
\kill
}{%
\end{tabbing}\end{list}}
% Uprogram* is like program*
\newenvironment{Uprogram*}{%
\samepage\programFontDeclaration% 
\begin{list}{}{}\item\catcode`\_=12\obeyspaces\obeytabs\obeycr}{%
\end{list}}

% comment text in italic roman
\newcommand{\programcommenttext}[1]{\textnormal{\textit{#1}}}
\newcommand{\programnewparagraph}{\\[0.5\baselineskip]}

% various program characters
\newcommand{\singlequote}{{\programFontDeclaration\textquoteright}}  %\char19
\newcommand{\doublequote}{{\programFontDeclaration"}}
\newcommand{\backquote}{{\programFontDeclaration\char18}}
\newcommand{\backwhack}{{\programFontDeclaration\textbackslash}} %\char`\\
\newcommand{\atsign}{{\programFontDeclaration\char`\@}}
\newcommand{\sharpsign}{{\programFontDeclaration\#}}
\newcommand{\verticalbar}{{\programFontDeclaration\textbar}}
\newcommand{\leftcurlybracket}{{\programFontDeclaration\{}\xspace}
\newcommand{\rightcurlybracket}{{\programFontDeclaration\}}\xspace}

% literate programming stuff
\newcommand{\OMITTEDREF}[1]{\mbox{$\langle$}{\em #1\/}\mbox{$\rangle$}}
\newcommand{\DEFINEIT}[1]{\mbox{$\langle$}{\em #1\/}\mbox{$\rangle ~~ \equiv$}}
\newcommand{\ADDTODEF}[1]{\mbox{$\langle$}{\em #1\/}\mbox{$\rangle ~~ +\equiv$}}
\newcommand{\WHEREUSED}[1]{{\footnotesize This code is used in #1.}}
\newcommand{\SEEALSO}[1]{{\footnotesize See also #1.}}

% program text within regular paragraphs, with hyphenation
\newcommand{\code}[1]{{\programFontDeclaration\hyphenchar\font=`-\relax {#1}}}
