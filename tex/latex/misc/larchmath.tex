% $Id: larchmath.tex,v 1.3 1997/07/28 18:57:53 leavens Exp $
% Also a bunch of math symbols to go with larch.sty, from Jim Horning

\newcommand{\join}{\Join}
\mathrename{\forall}{\mathforall}
\mathrename{\exists}{\mathexists}
\mathrename{\in}{\mathin}
\mathrename{\notin}{\mathnotin}
\mathrename{\ominus}{\mathominus}
\mathrename{\subset}{\mathsubset}
\mathrename{\subseteq}{\mathsubseteq}
\mathrename{\supset}{\mathsupset}
\mathrename{\supseteq}{\mathsupseteq}
\mathrename{\neq}{\mathneq}
\mathrename{\circ}{\mathcirc}
\mathrename{\odot}{\mathodot}
\mathrename{\oplus}{\mathoplus}
\mathrename{\equiv}{\mathequiv}
\mathrename{\Join}{\mathJoin}
\mathrename{\langle}{\mathlangle}
\mathrename{\rangle}{\mathrangle}
\newcommand{\<}{\mathlangle}          % added by Gary Leavens
\renewcommand{\>}{\mathrangle}        % added by Gary Leavens
\mathrename{\<}{\oldmathleftangle}    % added by Gary Leavens
\mathrename{\>}{\oldmathrightangle}   % added by Gary Leavens
\mathrename{\bot}{\mathbot}
\mathrename{\top}{\mathtop}
\mathrename{\times}{\mathtimes}

\newcommand{\inv}{^{-1}}
\mathrename{\inv}{\mathinv}
\newcommand{\lpand}{$\wedge$}
\newcommand{\lpor}{$\vee$}
\newcommand{\lpnot}{$\not$}
\newcommand{\lpimplies}{$\Rightarrow$}
\newcommand{\pre}{{\it pre}}
\mathrename{\pre}{\mathpre}
\newcommand{\post}{{\it post}}
\mathrename{\post}{\mathpost}
\newcommand{\reqP}{{\it reqP}}
\mathrename{\reqP}{\mathreqP}
\newcommand{\lpequiv}{\Leftrightarrow}
\mathrename{\lpequiv}{\mathlequiv}
\newcommand{\lequiv}{\Leftrightarrow} % For debugging chapter
\mathrename{\lequiv}{\mathlequiv}     % For debugging chapter
\newcommand{\I}{\cap}
\mathrename{\I}{\mathI}
\newcommand{\U}{\cup}
\mathrename{\U}{\mathU}
\newcommand{\rel}{\diamond}
\mathrename{\rel}{\mathdiamond}
\newcommand{\modList}{{\it modList}}
\mathrename{\modList}{\mathmodList}
\newcommand{\ensP}{{\it ensP}}
\mathrename{\ensP}{\mathensP}
\newcommand{\terminates}{{\it terminates}}
\mathrename{\terminates}{\mathterminates}
\newcommand{\modP}{{\it modP}}
\mathrename{\modP}{\mathmodP}
\newcommand{\gd}{{\tt =>}}            % For LM3 guard punctuation
\mathrename{\gd}{\mathgd}
\newcommand{\ra}{{\tt -}\hspace{-.1em}{\tt >}\hspace{-.5em}}  % For LCL ->
\mathrename{\ra}{\mathra}
\newcommand{\any}{^\bullet}
\mathrename{\any}{\mathany}
\newcommand{\precat}{\dashv}
\mathrename{\precat}{\mathprecat}
\newcommand{\postcat}{\vdash}
\mathrename{\postcat}{\mathpostcat}
\newcommand{\hatt}{{\tt \char'136}}
\mathrename{\hatt}{\mathhatt}
\newcommand{\qb}{{\tt \char'134}}
\mathrename{\qb}{\mathqb}
\newcommand{\superplus}{^\f{+}}
\mathrename{\superplus}{\mathsuperplus}
\newcommand{\superstar}{^\f{*}}
\mathrename{\superstar}{\mathsuperstar}
\newcommand{\glb}{\sqcap}
\mathrename{\glb}{\mathglb}
\newcommand{\lub}{\sqcup}
\mathrename{\lub}{\mathlub}
\newcommand{\append}{\vdash}
\mathrename{\append}{\mathappend}
\newcommand{\intersect}{\cap}
\mathrename{\intersect}{\mathintersect}
\newcommand{\plus}{\^{+}}
\mathrename{\plus}{\mathplus}
\newcommand{\prepend}{\dashv}
\mathrename{\prepend}{\mathprepend}
\newcommand{\union}{\cup}
\mathrename{\union}{\mathunion}



