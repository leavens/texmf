% Copyright 1989 by Norman Ramsey, Odyssey Research Associates
% To be used for research purposes only
% For more information, see file COPYRIGHT

%\let\plainmark=\mark
%\def\mark#1{\message{(Marking ``#1'')}\plainmark{#1}}

% standard macros for WEB listings (in addition to PLAIN.TEX)
% rename some old favorites
\let\amp=\&
\let\SS=\S
\let\PP=\P
\let\em=\it % compatibility with latex

\newif\iftwoside\twosidefalse
\parskip 0pt % .1pt plus 0.1pt mins 0.1pt % no stretch between paragraphs
\parindent 1em % for paragraphs and for the first line of Pascal text

\font\eightrm=cmr8
\font\sc=cmcsc10
\let\mainfont=\tenrm
\font\titlefont=cmr7 scaled\magstep4 % title on the contents page
\font\ttitlefont=cmtt10 scaled\magstep2 % typewriter type in title
\font\tentex=cmtex10 % TeX extended character set (used in strings)
\let\idfont\it
\let\reservedfont\bf

\def\today{\ifcase\month\or
  January\or February\or March\or April\or May\or June\or
  July\or August\or September\or October\or November\or December\fi
  \space\number\day, \number\year}

\def\\#1{\leavevmode\hbox{\idfont#1\/\kern.05em}} % italic type for identifiers
\def\|#1{\leavevmode\hbox{$#1$}} % one-letter identifiers look better this way
\def\&#1{\leavevmode\hbox{\reservedfont#1\/}} % boldface type for reserved words
\def\.#1{\leavevmode\hbox{\tentex % typewriter type for strings
  \let\\=\BS % backslash in a string
  \let\'=\RQ % right quote in a string
  \let\`=\LQ % left quote in a string
  \let\{=\LB % left brace in a string
  \let\}=\RB % right brace in a string
  \let\~=\TL % tilde in a string
  \let\ =\SP % space in a string
  \let\_=\UL % underline in a string
  \let\&=\AM % ampersand in a string
  #1}}
\def\#{\hbox{\tt\char`\#}} % parameter sign
\def\${\hbox{\tt\char`\$}} % dollar sign
\def\%{\hbox{\tt\char`\%}} % percent sign
\def\^{\ifmmode\mathchar"222 \else\char`^ \fi} % pointer or hat
% circumflex accents can be obtained from \^^D instead of \^
\def\AT!{@} % at sign for control text
\def\@{@} % at sign in strings
% EVERY WEAVE MUST DEFINE \? WHERE ? IS THE AT SIGN!!!!

% Macros to surround text in |...|
\def\CD{\relax\ifmmode\let\DC\egroup\hbox\bgroup\else\let\DC\relax\fi}
\let\DC=\relax

\chardef\AM=`\& % ampersand character in a string
\chardef\BS=`\\ % backslash in a string
\chardef\LB=`\{ % left brace in a string
\def\LQ{{\tt\char'22}} % left quote in a string
\chardef\RB=`\} % right brace in a string
\def\RQ{{\tt\char'23}} % right quote in a string
\def\SP{{\tt\char`\ }} % (visible) space in a string
\chardef\TL=`\~ % tilde in a string
\chardef\UL=`\_ % underline character in a string

\newbox\bak \setbox\bak=\hbox to -1em{} % backspace one em
\newbox\bakk\setbox\bakk=\hbox to -2em{} % backspace two ems

\newcount\ind % current indentation in ems
\def\0{\ifmmode\ifinner$\par % forced break
  \hangindent\ind em\noindent\kern\ind em\ignorespaces$\fi
  \else\par % forced break
  \hangindent\ind em\noindent\kern\ind em\ignorespaces\fi}
\def\1{\global\advance\ind by1\hangindent\ind em} % indent one more notch
\def\2{\global\advance\ind by-1} % indent one less notch
\def\3#1{\hfil\penalty#10\hfilneg} % optional break within a statement
\def\4{\copy\bak} % backspace one notch
\def\5{\hfil\penalty-1\hfilneg\kern2.5em\copy\bakk\ignorespaces}% optional break
%\def\6{\ifmmode\else\par % forced break with no indentation
%  \hangindent\ind em\noindent\kern\ind em\copy\bakk\ignorespaces\fi}
\def\6{\ifmmode\else\par % forced break with no indentation
  \hangindent\ind em\startline\ignorespaces\fi}
\def\7{\Y\6} % forced break and a little extra space
\def\8{\unskip} % no indentation--works only in code, not in |...|
\def\startline{\noindent
  \count255=\ind\advance\count255by-2
  \hskip\count255 em}

\let\yskip=\smallskip
\def\note#1#2.{\Y\noindent{\hangindent2em\baselineskip10pt\eightrm#1 #2.\par}}
\def\lapstar{\rlap{*}}
\def\startsection{\Q\noindent{\let\*=\lapstar\bf\modstar.\quad}}
\def\defin#1{\global\advance\ind by 2 \1\&{#1 }} % begin `define' or `format'
\def\A{\note{See also}} % cross-reference for multiply defined section names
\def\B{\mathopen{\.{@\commentbegin}}} % begin controlled comment
\def\C#1{\ifmmode\gdef\XX{\null$\null}\else\gdef\XX{}\fi % C comments
  \XX\hfil\penalty-1\hfilneg\quad
	$\commentbegin\,${#1}$\,\commentend$\XX}
\def\D{\defin{define}} % macro definition
\def\F{\defin{format}} % format definition
\def\J{\.{@\&}} % TANGLE's join operation
\outer\def\M#1.{\MN#1.\ifon\vfil\penalty-100\vfilneg % beginning of section
  \vskip12ptminus3pt\startsection\ignorespaces}
\outer\def\N#1.#2.{\MN#1.\headcheck#2\headcheck
  \edef\rhead{\uppercase{\ignorespaces\themodtitle}} % define running headline
  \message{*\modno} % progress report
  \edef\next{\write\cont{\thetocskip
	\Z{\theopen\relax
		\themodtitle}{\modno}{\noexpand\the\pageno}}}\next % to contents file
  \ifon\startsection{\bf\ignorespaces\themodtitle.\quad}\ignorespaces}
\def\MN#1.{\par % common code for \M, \N
  {\xdef\modstar{#1}\let\*=\empty\xdef\modno{#1}}
  \ifx\modno\modstar \onmaybe \else\ontrue \fi \mark{\modno}}
\def\O#1{% octal, hex or decimal constant
  {\def\?{\kern.2em}%
  \def\${\ell}% long constant
  \def\_{\cdot 10^{\aftergroup}}% power of ten
  \def\~{\hbox{\rm\char'23\kern-.2em\it\aftergroup\?\aftergroup}}% octal
  \def\^{\hbox{\rm\char"7D\tt\aftergroup}}% double quotes for hex constant
  #1}}
\def\P{\rightskip=0pt plus 100pt minus 10pt % go into Pascal mode
  \sfcode`;=3000
  \pretolerance 10000
  \hyphenpenalty 10000 \exhyphenpenalty 10000
  \global\ind=2 \1\startline}%\ \unskip}
\def\Q{\rightskip=0pt % get out of Pascal mode
  \sfcode`;=1500 \pretolerance 200 \hyphenpenalty 50 \exhyphenpenalty 50 }
\def\T{\mathclose{\.{@\commentend}}} % terminate controlled comment
\def\U{\note{This code is used in}} % cross-reference for uses of sections
\def\X#1:#2\X{\ifmmode\gdef\XX{\null$\null}\else\gdef\XX{}\fi % section name
  \XX$\langle\,$#2{\eightrm\kern.5em#1}$\,\rangle$\XX}
\def\XF{\begingroup\catcode`\_=12 \doXF}
\def\doXF#1:#2\XF{{\ifmmode\gdef\XX{\null$\null}\else\gdef\XX{}\fi
  \XX{\tt(#2{\eightrm\kern.5em#1})}\XX} \endgroup}
\def\Y{\par\yskip}
\def\){\hbox{\.{@\$}}} % sign for string pool check sum
\def\]{\hbox{\.{@\\}}} % sign for forced line break
\def\=#1{\kern2pt\hbox{\vrule\vtop{\vbox{\hrule
        \hbox{\strut\kern2pt\.{#1}\kern2pt}}
      \hrule}\vrule}\kern2pt} % verbatim string
\let\~=\ignorespaces
\let\*=*


\def\DO{\hbox{\sl\char'044}} % slant dollar sign
\let\G=\ge % greater than or equal sign
\def\H{{\rm\char'136}} % hat
\let\I=\ne % unequal sign
\let\K=\gets % left arrow
\let\L=\le % less than or equal sign
\let\R=\lnot % logical not
\let\S=\equiv % equivalence sign
\let\TI\sim % tilde
\let\V=\lor % logical or
\let\W=\land % logical and
\let\Z=\let % now you can \send the control sequence \Z


\def\onmaybe{\let\ifon=\maybe} \let\maybe=\iftrue
\newif\ifon \newif\iftitle \newif\ifpagesaved
\def\lheader{\mainfont\the\pageno\eightrm\qquad\rhead\hfill\title\qquad
  \tensy x\mainfont\topmark} % top line on left-hand pages
\def\rheader{\tensy x\mainfont\topmark\eightrm\qquad\title\hfill\rhead
  \qquad\mainfont\the\pageno} % top line on right-hand pages
\def\lfooter{\hfil} % bottom line on left-hand-pages
\def\rfooter{\hfil} % bottom line on left-hand-pages
\def\page{\box255 }
\def\normaloutput#1#2#3#4#5{%
%\message{(At start top, first, and bottom marks: \topmark, \firstmark, \botmark)}%
\shipout\vbox{
  \iftwoside\else\ifodd\pageno\hoffset=\pageshift\fi\fi
  \vbox to\fullpageheight{
     \iftitle
     \else\hbox to\pagewidth{\vbox to10pt{}%
   	\ifodd\pageno#3\else
		\iftwoside#2\else#3\fi
	\fi}%
     \fi
     \vfill#1% parameter #1 is the page itself
     \iftitle\global\titlefalse
     \else\baselineskip=24pt\hbox to\pagewidth{\strut % see TeXbook p256
	\ifodd\pageno#5\else
		\iftwoside#4\else#5\fi
	\fi}%
     \fi
  }%
}%
\global\advance\pageno by1
%\message{(At end top, first, and bottom marks: \topmark, \firstmark, \botmark)}%
}


\def\rhead{\.{WEB} OUTPUT} % this running head is reset by starred sections
\def\title{} % an optional title can be set by the user
\def\topofcontents{\centerline{\titlefont\title}
  \vfill} % this material will start the table of contents page
\def\botofcontents{\vfill} % this material will end the table of contents page
\def\contentspagenumber{0} % default page number for table of contents
\newdimen\pagewidth \pagewidth=6.5in % the width of each page
\newdimen\pageheight \pageheight=8.4in % the height of each page
\newdimen\fullpageheight \fullpageheight=9in % page height including
					     % headlines and footlines
\newdimen\pageshift \pageshift=0in % shift righthand pages wrt lefthand ones
\catcode`\@=11 % make at letter
\def\m@g{\mag=\count@\pagewidth=6.5truein\pageheight=8.4truein
  \fullpageheight=9truein\setpage}
\catcode`\@=12 % make at other
\def\setpage{\hsize\pagewidth\vsize\pageheight} % use after changing page size

\edef\contentsfile{\jobname.toc } % file that gets table of contents info
\def\readcontents{\expandafter\input \contentsfile}

\newwrite\cont
\output{\setbox0=\page % the first page is garbage
  \openout\cont=\contentsfile
  \write\cont{\string\catcode`\string\@=11}% a hack to make contents
	  				   % take stuff in \.{---}
  \global\output{\normaloutput\page\lheader\rheader\lfooter\rfooter}}
\setpage
\vbox to \vsize{} % the first \topmark won't be null
% Delete as per bug report from kademan@stat.wisc.edu
% Causes headings on first page to be missing
% \eject

\def\ch{\note{The following sections were changed by the change file:}
  \let\*=\relax}
\newbox\sbox % saved box preceding the index
\newbox\lbox % lefthand column in the index
\def\inx{\par\vskip6pt plus 1fil % we are beginning the index
  \write\cont{} % ensure that the contents file isn't empty
  \closeout\cont % the contents information has been fully gathered
  \output{\ifpagesaved\normaloutput{\box\sbox}\lheader\rheader
		\lfooter\rfooter\fi
    \global\setbox\sbox=\page \global\pagesavedtrue}
  \pagesavedfalse \eject % eject the page-so-far and predecessors
  \setbox\sbox\vbox{\unvbox\sbox} % take it out of its box
  \vsize=\pageheight \advance\vsize by -\ht\sbox % the remaining height
  \hsize=.5\pagewidth \advance\hsize by -10pt
    % column width for the index (20pt between cols)
  \parfillskip 0pt plus .6\hsize % try to avoid almost empty lines
  \def\lr{L} % this tells whether the left or right column is next
  \output{\if L\lr\global\setbox\lbox=\page \gdef\lr{R}
    \else\normaloutput{\vbox to\pageheight{\box\sbox\vss
        \hbox to\pagewidth{\box\lbox\hfil\page}}}\lheader\rheader
	\lfooter\rfooter
    \global\vsize\pageheight\gdef\lr{L}\global\pagesavedfalse\fi}
  \message{Index:}
  \parskip 0pt plus .5pt
  \outer\def\:##1, {\par\hangindent2em\noindent##1:\kern1em} % index entry
  \def\[##1]{$\underline{##1}$} % underlined index item
  \rm \rightskip0pt plus 2.5em \tolerance 10000 \let\*=\lapstar
  \hyphenpenalty 10000 \parindent0pt}
\def\fin{\par\vfill\eject % this is done when we are ending the index
  \ifpagesaved\null\vfill\eject\fi % output a null index column
  \if L\lr\else\null\vfill\eject\fi % finish the current page
  \parfillskip 0pt plus 1fil
  \def\rhead{NAMES OF THE SECTIONS}
  \message{Section names:}
  \output{\normaloutput\page\lheader\rheader\lfooter\rfooter}
  \setpage
  \def\note##1##2.{\quad{\eightrm##1 ##2.}}
  \def\U{\note{Used in}} % cross-reference for uses of sections
  \def\:{\par\hangindent 2em}\let\*=*}
\def\con{\par\vfill\eject % finish the section names
  \rightskip 0pt \hyphenpenalty 50 \tolerance 200
  \setpage
  \output{\normaloutput\page\lheader\rheader\lfooter\rfooter}
  \titletrue % prepare to output the table of contents
  \pageno=\contentspagenumber \def\rhead{TABLE OF CONTENTS}
  \message{Table of contents:}
  \topofcontents
  \line{\hfil Section\hbox to3em{\hss Page}}
  \def\Z##1##2##3{\line{{\ignorespaces##1}
    \leaders\hbox to .5em{.\hfil}\hfil\ ##2\hbox to3em{\hss##3}}}
  \readcontents\relax % read the contents info
  \botofcontents \end} % print the contents page(s) and terminate
%\tracingstats1 % temporary (during development)


\def\vert{|}
		
%%% this stuff is to allow inital =,1,2,3,4 in starred modules
%%% 	= means ``part'', don't skip page
%%%	normal starred module is 0
%%%		1,2,3,4 are submodules, and are indented
%%%
%%%	@*=	bold name in table of contents
%%%		causes page eject
%%%		suppresses page eject following
%%%
%%%	@*1,2	first level of indentation
%%%	@*3,4	second level of indentation
%%%
%%%     @*1,3	cause page eject
%%%     @*2,4	don't cause page eject
%%%


\newif\ifcancel\cancelfalse
\catcode`\@=11
\def\ifnextchar#1#2#3{\let\@tempe=#1\def\@tempa{#2}\def\@tempb{#3}\@ifnch}
\def\@ifnch{\ifx \@tempc \@tempe\let\@tempd\@tempa\else\let\@tempd\@tempb\fi
      \@tempd}
\def\makethechar#1{\let\@tempc=#1}
\catcode`\@=12

\def\headcheck#1#2\headcheck{%
	\makethechar{#1}%
	\def\theskipper{\vfil\penalty-100\vfilneg\vskip12ptminus3pt}%
				% skip before new module
	\def\theopen{}% opening skip in toc entry
	\def\thetocskip{}% vertical skip before toc entry
	\def\themodtitle{{#2}}
	\ifnextchar={%
		\def\theskipper{\vfil\eject}%
		\canceltrue
		\def\theopen{\bf}%
		\def\thetocskip{\vskip3ptplus1in\penalty-100
			\vskip0ptplus-1in}%
	}{\ifnextchar1{%
		\cancelfalse
		\def\theskipper{\vfil\eject}%
		\def\theopen{\hskip2em}%
	}{\ifnextchar2{%
		\cancelfalse
		\def\theopen{\hskip2em}%
	}{\ifnextchar3{%
		\cancelfalse
		\def\theskipper{\vfil\eject}%
		\def\theopen{\hskip4em}%
	}{\ifnextchar4{%
		\cancelfalse
		\def\theopen{\hskip4em}%
	}{% else 
		\ifcancel\else
			\def\theskipper{\vfil\eject}%
		\fi
		\cancelfalse
		\def\themodtitle{#1{#2}}%
	}}}}}%
	\theskipper
}


%%%%%%%% for verbatim quoting of code
% The following are copied from manmanc.tex and are taken from p421 of
% the TeXbook  ... modified to \verbatim...\endverbatim
% macros for verbatim scanning
\chardef\other=12
\def\ttverbatim{\begingroup
  \catcode`\|=\other
  \catcode`\\=\other
  \catcode`\{=\other
  \catcode`\}=\other
  \catcode`\$=\other
  \catcode`\&=\other
  \catcode`\#=\other
  \catcode`\%=\other
  \catcode`\~=\other
  \catcode`\_=\other
  \catcode`\^=\other
  \obeyspaces \obeylines \tt}
{\obeyspaces\global\let =\ } % from texbook, p 381


%\outer\def\verbatim{$$\let\par=\endgraf \ttverbatim \parskip=0pt
%  \catcode`\|=0 \rightskip-5pc \ttfinish}
%{\catcode`\|=0 |catcode`|\=\other % | is temporary escape character
%  |obeylines % end of line is active
%  |gdef|ttfinish#1^^M#2\endverbatim{#1|vbox{#2}|endgroup$$}}

\outer\def\verbatimcode{\par\ttverbatim\leftskip=2em\parskip=0pt
  \ttfinishcode}
{\catcode`\|=0 |catcode`|\=\other % | is temporary escape character
  |obeylines % end of line is active
  |gdef|ttfinishcode#1^^M#2\endverbatimcode{#1|vbox{#2}|endgroup}}


% end of manmac stuff

