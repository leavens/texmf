% local.tex -- released 27 October 1988
% Copyright(c) 1988 by Leslie Lamport
%   for LaTeX version 2.09
%
% This file is used to produce a Local Guide for LaTeX users containing
% information specific to a site plus errors and omissions from the LaTeX
% manual (published by Addison-Wesley).
%
% The installer of LaTeX at a site is responsible for customizing this
% document and providing copies for users.  He will have to read the
% text of this file CAREFULLY to see what must be added, removed, and
% changed.  

% The \contact command is defined to generate the name of the person to
% whom questions should be sent.  This should be someone at the site.
% Most users' questions are easily answered by anyone slightly familiar
% with LaTeX or TeX. Don't bother anyone at another site with questions
% that can be answered locally.

\documentclass{article}

\newcommand{\contact}{the Systems Support Group}

\newcommand{\BibTeX}{{\rm B\kern-.05em{\sc i\kern-.025em b}\kern-.08em
    T\kern-.1667em\lower.7ex\hbox{E}\kern-.125emX}}

\newcommand{\SLiTeX}{{\rm S\kern-.06em{\sc l\kern-.035emi}\kern-.06em T\kern
   -.1667em\lower.7ex\hbox{E}\kern-.125emX}}


\newcommand\bs{\char '134 }   % A backslash character for \tt font
\newcommand{\lb}{\char '173 } % A left brace character for \tt font
\newcommand{\rb}{\char '175 } % A right brace character for \tt font

\title{Using \LaTeX\ at ISU CS}

\author{Gary T. Leavens \& Dave Buckley}

\date{14 February 1996\\              % Keep this date current
For \LaTeX\ 2e} 

\begin{document}

\maketitle

\tableofcontents

\newpage

\LaTeX\ runs on a variety of computers at many different sites.  This
document tells you how to use \LaTeX\ on the CS department
computers, such as {\tt flash}, {\tt stimpy} and {\tt ren} at ISU.
(A few notes about running \LaTeX\ on Project Vincent are included,
but we cannot be responsible for documenting their system.)
It is not about \LaTeX\ itself, which is described by
the manual---{\em \LaTeX: A Document Preparation System\/} (second edition),
published by Addison-Wesley, available at fine book stores everywhere.

If you have a question that you can't answer by reading the manual and
this document, ask \contact.  They should also be informed of any
possible \LaTeX\ bugs or undocumented anomalies.


\section{Getting Started}

\subsection{Running a Sample File} \label{sec:sample}

Before preparing your own documents, you may want to get acquainted
with \LaTeX\ by running it on a sample input file.  First make your own
copy of the file \mbox{\tt sample2e.tex} by typing the following
command:
\begin{verbatim}
     cp /usr/local/lib/tex/inputs/sample2e.tex .
\end{verbatim}
(You must type the space followed by the period at the end.  This
and all other HP-UX commands are ended by typing {\em return}.)
A copy of the file \mbox{\tt sample.tex} is now in your current
directory; you can edit it just like any other file.  If you destroy or
mess up your copy, typing the above command again gets you a fresh
one.

(On Project Vincent, execute the following commands instead.)
\begin{verbatim}
    add tex
    cp /home/tex/lib/tex/macros/sample2e.tex .
\end{verbatim}

Next, run \LaTeX\ on the file \mbox{\tt sample2e.tex} by typing:
\begin{verbatim}
     latex sample2e
\end{verbatim}
When \LaTeX\ has finished, it will have produced the file \mbox{\tt
sample2e.dvi} in your directory.  On HP-UX you can print this file
by typing the commands
\begin{verbatim}
     dvips sample2e.dvi
\end{verbatim}
The output will be produced on the printer lw2, in the second floor
terminal room.

To print on a CS department printer from Project Vincent, type
\begin{verbatim}
     add cs
     dvips sample2e.dvi | cslpr -Plw2
\end{verbatim}
See the manual page for {\tt cslpr} for other options.

After your output has been printed, you can delete \mbox{\tt
sample.dvi} by typing
\begin{verbatim}
     rm sample2e.dvi 
\end{verbatim}

\subsection{Preparing and Running \LaTeX\ on Your Own Files}

You must use a text editor to prepare an input file for \LaTeX.
The easiest way to start learning about \LaTeX\ is by 
examining the file \mbox{\tt small2e.tex} with your text editor.
Your can get a copy by typing
\begin{verbatim}
     cp /usr/local/lib/tex/inputs/small2e.tex .
\end{verbatim}

After you have prepared your file, whose name should have the extension
{\tt tex}, you must run it through \LaTeX\ and print the output.
Follow the instructions in Section~\ref{sec:sample}, except substitute
the first name of your file for ``\mbox{\tt sample2e}''.  Remember to
save disk space by deleting the {\tt dvi} file after printing the
output.


%List the text editors, available, and any special features they have
%for producing \LaTeX\ input.  Explain how the various text editors
%could cause bad characters to appear in the input file that would
%generate the
%\begin{verbatim}
%! Text line contains an invalid character.
%\end{verbatim}
%error.

The {\tt emacs} editor has special support for \LaTeX.
Put the characters
\begin{verbatim}
    % -*- LaTeX -*-
\end{verbatim}
in one of the first lines of your file to get {\tt emacs} to recognize your
file as a \LaTeX\ input file.
In {\tt emacs} you can also type {\tt M-x latex-mode}
(that's emacs jargon for {\em Escape\/} then ``x'' followed by the characters
``latex-mode'')
to turn on \LaTeX\ support.
In latex-mode,
the command {\tt M-x validate-TeX-buffer} performs a sanity check
on a \LaTeX\ buffer.

If you want to stop \LaTeX\ in the middle of its execution, perhaps
because it is printing a seemingly unending string of uninformative
error messages, type {\em Control-C\/} (press $C$ while holding down
the key labeled {\em CTRL\/}).  This will make \LaTeX\ stop as if it
had encountered an ordinary error, and you can return to HP-UX command
level by typing {\tt X}, as described in the manual.  If typing {\em
Control-C\/} doesn't work, typing {\em Control-Z\/} will get you
immediately to HP-UX command level, but this will leave a stopped job
hanging around.  A stopped job won't hurt anything and will disappear
when you log out, but it forces you to type two successive \mbox{\tt
logout} commands to log out.

To use the {\em spell\/} program for finding spelling errors in a
\LaTeX\ input file named \mbox{\tt myfile.tex}, type the following
command:
%%% no detex or delatex here?
\begin{verbatim}
     delatex myfile.tex | spell
\end{verbatim}
This will type a list of possibly misspelled words on your terminal.
If you'd rather have the output written to a file named \mbox{\tt
foo.bar}, type
\begin{verbatim}
     delatex myfile.tex | spell >foo.bar
\end{verbatim}

\section{Carrying On}

\subsection{\LaTeX\ on HP-UX} \label{sec:op-system}

The only special problems in using \LaTeX\ caused by the HP-UX
operating system involve the way HP-UX handles files.  The first
problem arises because, when a program starts to write a file, HP-UX
destroys the previous version of that file.  Thus, if an error forces
you to stop \LaTeX\ prematurely (by typing {\em Control-C\/} or {\em
Control-Z\/}), then the files that \LaTeX\ was writing are incomplete,
and the previous complete versions have been destroyed.  You probably
don't care about the output on the {\tt dvi} file, but, if you are
making a table of contents or using cross-referencing commands, then
\LaTeX\ also writes one or more {\em auxiliary files\/} that it reads
the next time it processes the same input file.  If the auxiliary files
are incomplete because \LaTeX\ was stopped before reaching the end of
its input file, then the table of contents and cross-references will be
incorrect the next time \LaTeX\ is run on the same input file.  You
will have to run \LaTeX\ a second time to get them right.  If you want
to avoid having to run \LaTeX\ twice after making an error---for
example, if your input is very long---then you should save copies of
these auxiliary files before running \LaTeX. An input file named
\mbox{\tt myfile.tex} and all the auxiliary files produced by \LaTeX\
from it are included in the HP-UX file specifier \mbox{\tt myfile.*}.
Use the HP-UX {\tt cp} command to save copies of these files.

The second problem in using \LaTeX\ on HP-UX involves the files that
\LaTeX\ reads.  The file whose name you type with HP-UX's {\tt latex}
command is called the {\em root file}.  In addition to reading the root
file, \LaTeX\ also reads the files specified by \hbox{\verb|\input|}
and \hbox{\verb|\include|} commands.  With the HP-UX directory system,
\LaTeX\ must know not only the names of these file but also on what
directories they are.  It will have no problem finding the correct
files if you follow two simple rules:
\begin{enumerate}
 \item Run \LaTeX\ from the directory containing the root file.
 \item Keep all files specified by \hbox{\verb|\input|} and 
      \hbox{\verb|\include|} commands in the same directory as the root
       file.
\end{enumerate}
If you follow these rules, you never have to type an HP-UX path
specifier when using \LaTeX.

You should never break the first rule, otherwise \LaTeX\ will have
trouble finding auxiliary files.  (To run \LaTeX\ on someone else's
file, copy the file to your directory.) If you break the second
rule, specifying a file from another directory in an
\hbox{\verb|\input|} or \hbox{\verb|\include|} command, you can use a
complete path name.  For example, to include the file \mbox{\tt hisfile.tex} 
from Jones' directory \hbox{\verb|/foo/bar|}, you can type
\begin{verbatim}
     \include{/udir/jones/foo/bar/hisfile}
\end{verbatim}
A \verb|~| character may not appear in the argument of an
\hbox{\verb|\input|} or \hbox{\verb|\include|} command, so you {\em
can't\/} use a file name such as \hbox{\verb|~jones/foo/bar/hisfile|}.

For people who don't like to obey rules, 
here is exactly how \LaTeX\ finds its
files.  The root file is found by HP-UX according to its usual rules.
\LaTeX's auxiliary files are read and written in the directory from
which it is run.  All file names specified in the \LaTeX\ input,
including the names of document-style ({\tt sty}) files specified by
the \hbox{\verb|\documentstyle|} command, are interpreted relative to
the directory from which \LaTeX\ is run.  If \LaTeX\ does not find a
file starting in this directory, it looks in the system directory
\begin{verbatim}
     /usr/local/lib/tex/inputs
\end{verbatim}
(On Project Vincent it is \hbox{\verb|/home/tex/lib/tex/inputs|}.
You can change the directories in
which \LaTeX\ looks for its input files by setting the environment
variable \mbox{\tt TEXINPUTS}.  Putting the command
\begin{verbatim}
     set ULLT=/usr/local/lib/tex
     setenv STDTEX ${ULLT}/inputs:/home/tex/lib/tex/macros:${ULLT}/macros
     setenv TEXINPUTS .:/home/leavens/include:/usr/unsup/tex:$STDTEX
\end{verbatim}
in your \mbox{\tt .login} file causes \LaTeX\ to look for files first
in the current directory, then in Leavens's {\tt include} directory,
then in the unsupported tex directory {\tt /usr/unsup/tex}, and
then in the standard system directories (for HP-UX, Vincent, and Suns).
You might want to do this if your name
is Leavens and you have your own personal document-style files in your
{\tt include} directory. 

\subsection{Document Styles, Classes and Packages}

There are several document styles, classes, and packages
at ISU that are not described in the manual (or in {\em The \LaTeX\ Companion}: 
\begin{itemize}
\item the \mbox{\tt isuabbr} style option for
locally defined abbreviations.

\item The \mbox{\tt isuletter} style for making letters.

\item The {\tt isudoc} style option that is a local variation of the
standard report style.

\item The {\tt isuthes} style for theses.
\end{itemize} 

There are also some style, class, and package
files in {\tt /usr/unsup/tex}.

\subsubsection{Letters} \label{sec:letters}

The \mbox{\tt letter} document style, described in the manual, should
be used for generating personal letters.  For generating letters with the
ISU letterhead, use the \mbox{\tt isuletter} style.

There have been problems in the past printing this;
you may have to adjust the margins.
See the manual page for {\tt dvips}.

\subsubsection{The {\tt isuthes} Style Option}

The easy way to find out about this is to ask your friends.
The brave can look at the file {\tt isuthes.doc} in the appropriate
directory (from the {\tt TEXINPUTS} variable defined above).


\subsection{Where the Files Are}

% must explain where the following files are:
%   small.tex, sample.tex, *.sty, *.doc, lablst.tex, idx.tex
%

All \LaTeX\ files mentioned in the manual, including the {\tt sty} and
{\tt doc} files, are in the directory \mbox{\tt /usr/local/lib/tex/inputs}
on HP-UX.  They are in the  \mbox{\tt /home/tex/lib/tex/macros} directory
on Project Vincent.


\subsection{Differences from the Manual}

All \LaTeX\ features described in the manual are provided by 
the implementation at ISU.

%Explain here any characters that can appear in input files other than
%the ones listed in Section 2.1.

%Tell if the \mbox{\tt log} file has an extension other than
%\mbox{\tt .log}.  Note: on TOPS-20, its extension is \mbox{\tt .lst}.

%Describe the sizes of disks and circles the are available.

%Don't forget to mention if the invisible fonts needed for \SLiTeX\
%color slides are unavailable.

\subsection{Using \BibTeX}

\BibTeX\ is a program for compiling a reference list for a document
from a bibliographic database.  It is run by typing
\begin{verbatim}
     bibtex myfile
\end{verbatim}
where \mbox{\tt myfile.tex} is the name of your \LaTeX\ input file.
This reads the file \mbox{\tt myfile.aux}, which was generated when you
ran \LaTeX\ on \mbox{\tt myfile.tex}, and produces the file \mbox{\tt
myfile.bbl}.  \BibTeX\ should be run from the directory containing
\mbox{\tt myfile.tex} (which should be the same directory from which
\LaTeX\ was run on that file).

If the {\tt bib} file is not in the same directory as the \LaTeX\ input
file---for example, if you're using someone else's {\tt bib}
file---then you must include a path as part of the file name specified
by the \hbox{\verb|\bibliography|} command.  A \verb|~| cannot appear
in the argument of a \hbox{\verb|\bibliography|} command, so you should
use a complete path name.  For example, the \LaTeX\ command
\begin{verbatim}
     \bibliography{/udir/jones/bibfiles/gnus}
\end{verbatim}
specifies the file \mbox{\tt gnus.bib} kept by Jones in his 
\mbox{\tt /bibfiles} directory.


There is now no formal provision for sharing bibliographic database
information, nor are there programs to assist in making your own {\tt
bib} files.
%  Suggestions for forming one or more common {\tt bib} files are welcome.

In addition to the bibliography styles described in the manual, there
is a {\tt ieeetr} style that formats entries in the style of the IEEE
transactions.
There are several additional styles in {\tt /usr/unsup/tex}.

In addition to the usual three-letter abbreviations for the months, the
following abbreviations are defined by the bibliography styles:
\begin{list}{}{\labelwidth 0pt \itemindent-.5\leftmargin
       \itemsep=2pt plus 1pt
       \let\makelabel\descriptionlabel}\it
\item[\tt acmcs] ACM Computing Surveys
\item[\tt acta] Acta Informatica
\item[\tt cacm] Communications of the ACM
\item[\tt ibmjrd] IBM Journal of Research and Development
\item[\tt ibmsj] IBM Systems Journal
\item[\tt ieeese] IEEE Transactions on Software Engineering
\item[\tt ieeetc] IEEE Transactions on Computers
\item[\tt ieeetcad]
 IEEE Transactions on Computer-Aided Design of Integrated Circuits
\item[\tt ipl] Information Processing Letters
\item[\tt jacm] Journal of the ACM
\item[\tt jcss] Journal of Computer and System Sciences
\item[\tt scp] Science of Computer Programming
\item[\tt sicomp] SIAM Journal on Computing
\item[\tt tocs] ACM Transactions on Computer Systems
\item[\tt tods] ACM Transactions on Database Systems
\item[\tt tog] ACM Transactions on Graphics
\item[\tt toms] ACM Transactions on Mathematical Software
\item[\tt toois] ACM Transactions on Office Information Systems
\item[\tt toplas] ACM Transactions on Programming Languages and Systems
\item[\tt tcs] Theoretical Computer Science
\end{list}

% Note: All styles should share the same set of abbreviations.

\subsection{Using {\em makeindex\/}} \label{sec:makeindex}

The {\em makeindex\/} program helps in making an index.  It is 
described in the manual page.

\subsection{Special Versions}

No foreign-language or other special versions of \LaTeX\
are currently available at ISU in the CS department.


\subsection{Including PostScript graphics}

On HP-UX,
PostScript graphics can be including in \LaTeX\ documents using the 
{\tt psfig} macro.
(To see how to do this on Project Vincent, use the {\tt olc}
command, and select {\tt answers}, then select {\tt TeX and LaTeX answers},
and then the appropriate menu item.)
The {\tt psfig} macro assumes that the PostScript file to be included has a
\verb|%%BoundingBox:| comment at the beginning of the file.
(Most PostScript output generated by modern software will contain this 
comment.)
Insert the line
\begin{verbatim}
     \input{psfig}
\end{verbatim}
at the beginning of your document.
Then, if you wish to include the file {\tt bogosity.ps}
in your document, you would
insert the following line in your \LaTeX\ document:
\begin{verbatim}
     \centerline{\psfig{file=bogosity.ps}}
\end{verbatim}
This includes the PostScript graphic described in the file
{\tt bogosity.ps} at the current point in the file, centering
it with regard to the left and right margins.

Other options to the \verb|\psfig| macro are {\tt height}, 
{\tt width} and {\tt scale}.
The {\tt height} and {\tt width} options specify the height
and width of the included graphic, and take as their arguments
any valid \LaTeX\ unit.
Thus, the line 
\begin{verbatim}
     \centerline{\psfig{file=bogosity.ps,width=2in,height=1in}}
\end{verbatim}
will cause {\tt psfig} to scale the graphic disproportionately,
so that it is twice as wide as it is tall.
The {\tt scale} option scales both the height and width of the 
included figure by the specified ratio.  
Thus, if you wanted to draw an included graphic at 80\% of 
its original size, the line
\begin{verbatim}
     \centerline{\psfig{file=bogosity.ps,scale=80}}
\end{verbatim}
will do the trick.

\end{document}
